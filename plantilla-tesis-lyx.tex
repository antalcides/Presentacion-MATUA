%% LyX 2.1.3 created this file.  For more info, see http://www.lyx.org/.
%% Do not edit unless you really know what you are doing.
\documentclass[spanish]{beamer}
\usepackage{mathptmx}
\usepackage[T1]{fontenc}
\usepackage[latin9]{inputenc}
\usepackage{amsmath}
\usepackage{amssymb}

\makeatletter
%%%%%%%%%%%%%%%%%%%%%%%%%%%%%% Textclass specific LaTeX commands.
 % this default might be overridden by plain title style
 \newcommand\makebeamertitle{\frame{\maketitle}}%
 % (ERT) argument for the TOC
 \AtBeginDocument{%
   \let\origtableofcontents=\tableofcontents
   \def\tableofcontents{\@ifnextchar[{\origtableofcontents}{\gobbletableofcontents}}
   \def\gobbletableofcontents#1{\origtableofcontents}
 }

%%%%%%%%%%%%%%%%%%%%%%%%%%%%%% User specified LaTeX commands.
\batchmode
\makeatletter
\def\input@path{{style/}{sections/}{pdf/}{logos/}}
\makeatother
%-------------------------------------------------------
% Inclusi\'{o}n de paquetes
%-------------------------------------------------------
%\setbeamertemplate{frametitle continuation}{\frametitle{\color{white}Title}}
\usepackage{lipsum-es}
%\usepackage[latin9]{inputenc}
%\usepackage[ansinew]{inputenc}
%--------------------------------
%
\usepackage{ifthen}
\usepackage[T1]{fontenc}
\usepackage[mathletters]{ucs}
%\usepackage[utf8x]{inputenc}
\usepackage[english,spanish,es-tabla]{babel}
\uselanguage{spanish}
\languagepath{spanish}
\usepackage{environ}
\usepackage{lmodern}
\usefonttheme[onlymath]{serif}
\usefonttheme{professionalfonts}
%\usepackage[spanish]{babel}
\usepackage{times}
%---------------------------------
\usepackage{amssymb,amsfonts,latexsym,cancel,stmaryrd}%
\usepackage[ruled,,vlined,lined,linesnumbered]{algorithm2e}
\usepackage{framed}
\usepackage{mathptmx}
\usepackage{helvet}
\usepackage[full]{textcomp}                        % Caracteres especiales como ' (recto)
\usepackage{amsmath}
\makeatletter
\let\@tmp\@xfloat
\usepackage{fixltx2e}
\let\@xfloat\@tmp
\makeatother
\makeatletter
\let\th@plain\relax
\makeatother
\theoremstyle{plain}
\let\openbox\relax
%\usepackage{ntheorem}
\let\proofname\relax
\let\proof\relax
\let\endproof\relax
\usepackage{amsthm}
%\usepackage{ddot}
%\usepackage{a4wide}
\usepackage{amsfonts}
\usepackage{epsfig}
%\usepackage{amscd}
\usepackage{longtable}
%\usepackage{latexcad}
\usepackage{fancybox}
\graphicspath{{ps/}{logos/}{figuras/}{sections/Figures/}}
%\listfiles
% COMANDOS PERSONALES   ----------------------------------------------

%----------- cargando el Tema Tesis
\DeclareGraphicsExtensions{.eps,.jpg,.png, .bmp}

\usetheme[]{Tesis}%%% opci\'{o}n para el tema outher.
%    progressstyle=fixedCircCnt,   % fixedCircCnt, movingCircCnt (Movimiento por defecto)

% Si desea cambiar los colores de los diversos elementos en el tema, edite y descomente las siguientes l\'{\i}neas

% Para cambiar los colores de la barra:
%\setbeamercolor{Tesis}{fg=red!20,bg=red}

% Para cambiar los colores de la estructura:
%\setbeamercolor{structure}{fg=red}

%Para cambiar el color del texto de la caja del titulo:
%\setbeamercolor{frametitle}{fg=blue}

%Para cambiar el color del texto y del fondo:
%\definecolor{azultext}{RGB}{43,93,156}

%-------------------------------------------------------
% Definiendo y redefiniendo comandos y entornos
%-------------------------------------------------------

% colores de los hiperlinks
\newcommand{\chref}[2]{
  \href{#1}{{\usebeamercolor[bg]{Feather}#2}}
}
%\setbeamertemplate{headline}[text line]{\insertsectionnavigationhorizontal{\paperwidth}{}{}}
%-------------------------------------------------------
% Informacio\'{o}n de la p\'{a}gina del titulo
%-------------------------------------------------------
%%% configuracion del tema %%%%%%%%%%%%%%%%%%%5
\makeatletter
  \definecolor{beamer@barcolor}{RGB}{21,46,128}%azul
  \definecolor{beamer@headercolor}{RGB}{226,107,59}%naranja
\makeatother
\newcommand{\cdefault}[4][named]{\begin{tikzpicture}
\fill[#2,draw=negro] (0,0) rectangle ++(2,1);
\node[below] at (1,0) {#2};
\node[below=4mm] at (1,0) {\tiny #3 \{#4\}};
\node[below=6mm] at (1,0) {\tiny #1};
\end{tikzpicture}}
%%%%%%%%%%%%%%%%%%%%%%%%%%%%%%%%%%%%%%%%%%%%%%%
\newsavebox\terminalbox
\lstnewenvironment{terminal}[1][]
  {\lstset{#1}\setbox\terminalbox=\vbox\bgroup\hsize=0.7\textwidth}
  {\egroup
   \tikzstyle{terminal} = [
    draw=white, text=white, font=courier, fill=blue!20, very thick,
    rectangle, inner sep=10pt, inner ysep=20pt
   ]
   \tikzstyle{terminalTitle} = [
     fill=red!20, text=white, font=\ttfamily, draw=white
   ]
   \begin{tikzpicture}
   \node [terminal] (box){\usebox{\terminalbox}};
   \node[terminalTitle, rounded corners, right=10pt] at (box.north west) {tty: /bin/bash};
   \end{tikzpicture}
}
%%%%%%%%%%------------------------------------%%%%%%%%%%
\newenvironment{slides}[1]
{\begin{frame}[fragile,allowframebreaks, environment=slides]{#1}}
{\end{frame}}

%%%%%%%%%%%-------------------------------%%%%%%%%%%%%%%
%------------------------------------
\titulogrado{Mag\'{\i}ster}
\director{Dr: Jorge Rodr\'{\i}guez\\ {\footnotesize email:jorge.jrodri@gmail.com}}
\institutedirector{Universidad del Atl\'antico}
\author[Ang\'{e}lica Arroyo]{Ang\'{e}lica Arroyo\\ {\footnotesize email: angelicarroyo1020@gmail.com}}
\institute{Universidad del Atl\'antico}
\title{Acotaci\'{o}n de Soluciones e Integrales de Darboux en un Sistema Particular del Circuito de Chua}
%\subtitle{\LaTeX \ una imprenta con estilo}
\date{Julio de 2016}
%\subtitle{Linguistics as a Window for Understanding the Brain}
%\titlegraphic{\includegraphics[width=1.5cm]{donald}}

%-------------------------------------------------------
% Cuerpo de la presentaci\'{o}n
%-------------------------------------------------------

\makeatother

\usepackage{babel}
\addto\shorthandsspanish{\spanishdeactivate{~<>}}

\begin{document}
\begin{frame}[plain,noframenumbering]
\titlepage
\end{frame}



\AtBeginSubsection[]{%
  \frame<beamer>{
    \frametitle{\'{I}ndice}
    \tableofcontents[currentsection,currentsubsection]
  }
}



%\beamerdefaultoverlayspecification{<+->}
\begin{frame}{Contenido}


\tableofcontents{}




\end{frame}

\section{Motivaci\'{o}n}


\subsection[Problema b\'{a}sico]{El problema b\'{a}sico que estudiamos}
\begin{frame}{Poner t\'{\i}tulos informativos}


\framesubtitle{Los subt\'{\i}tulos de los fotogramas son opcionales.}
\begin{itemize}
\item Usar Enumeraci\'{o}n{*} a discreci\'{o}n.


\pause{}

\item Usar oraciones muy cortas o frases cortas.


\pause{}

\item Estos solapados se crean usando el estilo Pausa.
\end{itemize}
\end{frame}

\begin{frame}{Poner t\'{\i}tulos informativos }

\begin{itemize}
\item<1-> Tambi\'{e}n se pueden usar estas especificaciones para crear solapados
\item<3-> Esto permite presentar cosas en cualquier orden
\item<2-> \'{E}sta se muestra en segundo lugar
\end{itemize}
\end{frame}

\begin{frame}{Poner t\'{\i}tulos informativos}

\begin{block}<1->{}

\begin{itemize}
\item Bloque sin t\'{\i}tulo.
\item Mostrado en todas las diapositivas.
\end{itemize}
\end{block}
\begin{exampleblock}<2->{T\'{\i}tulo de alg\'{u}n Bloque de ejemplo}

\begin{itemize}
\item $e^{i\pi}=-1$.
\item $e^{i\pi/2}=i$.
\end{itemize}
\end{exampleblock}
\end{frame}

\subsection{Trabajo previo}
\begin{frame}{Poner t\'{\i}tulos informativos}



\begin{example}<1->
En la primera diapositiva.
\end{example}



\begin{example}<2->
En la segunda diapositiva.
\end{example}

\end{frame}

\section{Nuestros resultados/Contribuci\'{o}n}


\subsection{Resultados principales}
\begin{frame}{Poner t\'{\i}tulos informativos}

\begin{theorem}
En la primera diapositiva.
\end{theorem}


\pause{}
\begin{corollary}
En la segunda diapositiva.
\end{corollary}

\end{frame}

\begin{frame}{Poner t\'{\i}tulos informativos }

\begin{columns}[t]


\column{5cm}
\begin{theorem}<1->
En la columna de la izquierda.
\end{theorem}


\column{5cm}
\begin{corollary}<2->
En la columna de la derecha.\\
Nueva l\'{\i}nea
\end{corollary}

\end{columns}

\end{frame}

\subsection{Ideas b\'{a}sicas para demostraciones/implementaciones}


\section*{Sumario}
\begin{frame}{Sumario}

\begin{itemize}
\item El \alert{primer mensaje principal }de la exposici\'{o}n en una o dos
l\'{\i}neas.
\item El \alert{segundo mensaje principal} de la exposici\'{o}n en una o dos
l\'{\i}neas.
\item Quiz\'{a}s un \alert{tercer mensaje}, pero no m\'{a}s.
\end{itemize}




\medskip{}

\begin{itemize}
\item Perspectiva

\begin{itemize}
\item Lo que no hemos hecho todav\'{\i}a.
\item Otras cosas pendientes.
\end{itemize}
\end{itemize}
\end{frame}
\appendix

\section*{Ap\'{e}ndice}


\subsection*{Lecturas complementarias}
\begin{frame}[allowframebreaks]{Lecturas complementarias}


\beamertemplatebookbibitems
\begin{thebibliography}{1}
\bibitem{Autor1990}A. Autor. \newblock\emph{Manual de Lo que sea}.\newblock
Editorial, 1990.\beamertemplatearticlebibitems

\bibitem{Alguien2002}S. Alguien.\newblock Sobre esto y aquello\emph{.}
\newblock\emph{Revista Esto y Aquello}. 2(1):50--100, 2000.\end{thebibliography}
\end{frame}

\end{document}
