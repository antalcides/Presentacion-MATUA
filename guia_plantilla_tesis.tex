%% LyX 2.1.3 created this file.  For more info, see http://www.lyx.org/.
%% Do not edit unless you really know what you are doing.
\documentclass[ruled,,vlined,lined,linesnumbered]{beamer}
\usepackage{mathptmx}
\usepackage{helvet}
\usepackage[T1]{fontenc}
\usepackage[utf8x]{inputenc}
\setcounter{secnumdepth}{3}
\setcounter{tocdepth}{3}
\usepackage{algorithm2e}
\usepackage{amsmath}
\usepackage{amssymb}
\usepackage{cancel}
\usepackage{stmaryrd}

\makeatletter
%%%%%%%%%%%%%%%%%%%%%%%%%%%%%% Textclass specific LaTeX commands.
 % this default might be overridden by plain title style
 \newcommand\makebeamertitle{\frame{\maketitle}}%
 % (ERT) argument for the TOC
 \AtBeginDocument{%
   \let\origtableofcontents=\tableofcontents
   \def\tableofcontents{\@ifnextchar[{\origtableofcontents}{\gobbletableofcontents}}
   \def\gobbletableofcontents#1{\origtableofcontents}
 }

\@ifundefined{date}{}{\date{}}
%%%%%%%%%%%%%%%%%%%%%%%%%%%%%% User specified LaTeX commands.
\batchmode

\def\input@path{{style/}{sections/}{pdf/}{logos/}}




%-------------------------------------------------------
% Inclusi\'on de paquetes
%-------------------------------------------------------
%\setbeamertemplate{frametitle continuation}{\frametitle{\color{white}Title}}
%\usepackage{lipsum-es}
%\usepackage[latin9]{inputenc}
%\usepackage[ansinew]{inputenc}
%--------------------------------
%
\usepackage{ifthen}
\usepackage{ucs}
\usepackage[english,spanish,es-tabla]{babel}
%\uselanguage{spanish}
%\languagepath{spanish}
\usepackage{environ}
\usefonttheme[onlymath]{serif}
\usefonttheme{professionalfonts}
%\usepackage[spanish]{babel}
\usepackage{times}




%---------------------------------
\usepackage{amsfonts}\usepackage{latexsym}\usepackage{framed}
%\linespread{1.05}
\usepackage[full]{textcomp}% Caracteres especiales como ' (recto)

\let\@tmp\@xfloat     
\usepackage{fixltx2e}
\let\@xfloat\@tmp                    


\let\th@plain\relax

\theoremstyle{plain}
\let\openbox\relax
%\usepackage{ntheorem}
\let\proofname\relax
\let\proof\relax
\let\endproof\relax
\usepackage{amsthm}
%\usepackage{ddot}
%\usepackage{a4wide}
\usepackage{amsfonts}
\usepackage{epsfig}
%\usepackage{amscd}
\usepackage{longtable}
%\usepackage{latexcad}
\usepackage{fancybox}
\graphicspath{{ps/}{logos/}{figuras/}{sections/Figures/}}
\listfiles
%%%% traducción de rotulos %%%%%%%%%%%%%%%%
\gdef\abstractname{Resumen}
\def\bibname{Bibliograf\'{\i}a}
\def\appendixname{\textcolor{myred}{\HUGE{Ap\'endice}}}
\def\contentsname{Contenido}
\def\listfigurename{\'Indice de Figuras}
\def\listtablename{\'Indice de Tablas}
\def\indexname{\scshape{\'Indice}}
\def\figurename{Figura}
\gdef\tablename{Tabla}
% COMANDOS PERSONALES   ----------------------------------------------
\newcommand{\R}{\mathbb{R}}
\newcommand{\Z}{\mathbb{Z}}
\newcommand{\Q}{\mathbb{Q}}
\newcommand{\N}{\mathbb{N}}
\newcommand{\I}{\mathbb{I}}
\newcommand{\raya}{\rule{2cm}{0.01cm}\\}
\newcommand{\ds}{\displaystyle}
\newcommand{\sen}{\mathop{\rm sen}\nolimits}
\newcommand{\senh}{\mathop{\rm senh}\nolimits}
\newcommand{\arcsen}{\mathop{\rm arcsen}\nolimits}
\newcommand{\arcsec}{\mathop{\rm arcsec}\nolimits}
%\def\sen{\mathop{\mbox{\normalfont sen}}\nolimits}
\def\max{\mathop{\mbox{\normalfont m\’ax}}}
\newcommand{\bc}{\begin{center}}
\newcommand{\ec}{\end{center}}
\newcommand{\be}{\begin{enumerate}}
\newcommand{\ee}{\end{enumerate}}
%----------- cargando el Tema Tesis
\DeclareGraphicsExtensions{.eps,.jpg,.png, .bmp}


\usetheme[fixedCircCnt]{Tesis}%%% opción para el tema .
%    progressstyle=fixedCircCnt,   % fixedCircCnt, movingCircCnt (Movimiento por defecto)
  
% Si desea cambiar los colores de los diversos elementos en el tema, edite y descomente las siguientes líneas

% Para cambiar los colores de la barra:
%\setbeamercolor{Tesis}{fg=red!20,bg=red}

% Para cambiar los colores de la estructura:
%\setbeamercolor{structure}{fg=red}

%Para cambiar el color del texto de la caja del titulo:
%\setbeamercolor{frametitle}{fg=blue}

%Para cambiar el color del texto y del fondo:
%\definecolor{azultext}{RGB}{43,93,156}

%-------------------------------------------------------
% Definiendo y redefiniendo comandos y entornos
%-------------------------------------------------------

% colores de los hiperlinks
\newcommand{\chref}[2]{
  \href{#1}{{\usebeamercolor[bg]{Feather}#2}}
}
%\setbeamertemplate{headline}[text line]{\insertsectionnavigationhorizontal{\paperwidth}{}{}}
%-------------------------------------------------------
% Informacioón de la página del titulo
%-------------------------------------------------------
%%% configuracion del tema %%%%%%%%%%%%%%%%%%%5

  \definecolor{beamer@barcolor}{RGB}{21,46,128}%azul
  \definecolor{beamer@headercolor}{RGB}{226,107,59}%naranja
 \newcommand{\cdefault}[4][named]{\begin{tikzpicture}
\fill[#2,draw=negro] (0,0) rectangle ++(2,1);
\node[below] at (1,0) {#2};
\node[below=4mm] at (1,0) {\tiny #3 \{#4\}};
\node[below=6mm] at (1,0) {\tiny #1};
\end{tikzpicture}}
%%%%%%%%%%%%%%%%%%%%%%%%%%%%%%%%%%%%%%%%%%%%%%%
\newsavebox\terminalbox
\lstnewenvironment{terminal}[1][]
  {\lstset{#1}\setbox\terminalbox=\vbox\bgroup\hsize=0.7\textwidth}
  {\egroup
   \tikzstyle{terminal} = [
    draw=white, text=white, font=courier, fill=blue!20, very thick,
    rectangle, inner sep=10pt, inner ysep=20pt
   ]
   \tikzstyle{terminalTitle} = [
     fill=red!20, text=white, font=\ttfamily, draw=white
   ]
   \begin{tikzpicture}
   \node [terminal] (box){\usebox{\terminalbox}};
   \node[terminalTitle, rounded corners, right=10pt] at (box.north west) {tty: /bin/bash};
   \end{tikzpicture}
}
%%%%%%%%%%------------------------------------%%%%%%%%%%
\newenvironment{slides}[1]{\begin{frame}[fragile,allowframebreaks, environment=slides]{#1}}{\end{frame}}

%%%%%%%%%%%-------------------------------%%%%%%%%%%%%%%
%------------------------------------
\autor[Antalcides Olivo]{Antalcides Olivo\\ {\footnotesize email: an@gmail}}
\institute{Universidad del Atl\'antico}
\titulo{Personalizando el tema {\tt Tesis} de {\tt Beamer}}
\subtitulo{\LaTeX \ una imprenta con estilo}

\titulographic{\includegraphics[width=2.5cm]{logos/2}}
\author[Antalcides Olivo]{Antalcides Olivo\\ {\footnotesize email: an@gmail}}
\title{Como realizar una presentación con Beamer}
\subtitle{\LaTeX \ una imprenta con estilo}
%-------------------------------------------------------
% Cuerpo de la presentación
%-------------------------------------------------------

\makeatother

\usepackage{babel}
\addto\shorthandsspanish{\spanishdeactivate{~<>}}

\begin{document}
\begin{frame}[plain,noframenumbering] 
\titulopage 
\end{frame}



\AtBeginSubsection[]{%
  \frame<beamer>{ 
    \frametitle{Índice}   
    \tableofcontents[currentsection,currentsubsection] 
  }
}
\begin{frame}{Contenido}


\tableofcontents{}




\end{frame}

\section{Diapositiva del Título}


\subsection{Opción Tesis}
\begin{frame}[fragile]{Diapositiva del título en la opción tesis}


Para crear la diapositiva del titulo se debe colocar en el preámbulo
del documento lo siguiente :

\begin{source}{Pre\'ambulo}{c1}
\titulo{Especialista}
\director{Dr: Alejandro Urieles\\ {\footnotesize email: al@gmail}}
\institutedirector{Universidad del Atl\'antico}
\author[Antalcides Olivo]{Antalcides Olivo\\ {\footnotesize email: an@gmail}}
\institute{Universidad del Atl\'antico}
\title{Como realizar tu defensa de tesis con Beamer}
\subtitle{\LaTeX \ una imprenta con estilo}
\date{}
\end{source}

Luego se crea un Frame para la diapositiva con el código:

\begin{source}{Titulo}{c2} 
\begin{frame}[plain,noframenumbering] 
\titlepage 
% \end{frame}  El % que ?está? antes de \end{frame} debe quitarlo 
\end{source}
Para cambiar el logo del director debe colocar en el preámbulo.
\begin{source}{}{Logo del Director}{c2a}
\titlegraphic{\includegraphics[scale=0.5]{2a}}
donde 2a el nombre del archivo del logo.
\end{source}
\end{frame}

\subsection{Opción Charla}
\begin{frame}[fragile]{Diapositiva del título en la opción: Charla }


En esta caso para crear la diapositiva del titulo se debe colocar
en el preámbulo del documento lo siguiente :

\begin{source}{Pre\'ambulo}{c3}
\autor[Antalcides Olivo]{Antalcides Olivo\\ {\footnotesize email: an@gmail}}
\institute{Universidad del Atl\'antico}
\titulo{Como realizar una presentación con Beamer}
\subtitulo{\LaTeX \ una imprenta con estilo}
\titulographic{\includegraphics[width=2.5cm]{logos/2}}
\author[Antalcides Olivo]{Antalcides Olivo\\ {\footnotesize email: an@gmail}}
\title{Como realizar una presentación con Beamer}
\subtitle{\LaTeX \ una imprenta con estilo}
\date{}
\end{source}

Luego se crea un Frame para la diapositiva con el código:

\begin{source}{Titulo}{c4} 
\begin{frame}[plain,noframenumbering] 
\titulopage 
 \end{frame}   
\end{source}
\end{frame}

\section{Diapositiva del contenido}
\begin{slides}{Diapositiva de la tabla de contenido }
Si quieres que el índice se muestre en cada ,   debes colocar
el siguiente código
\begin{source}{\'Indice al comienzo de cada subsecci\'on}{c5}
\AtBeginSubsection[]{%
  \frame<beamer>{ 
    \frametitle{Índice}   
    \tableofcontents[currentsection,currentsubsection] 
  }
}
\end{source}
Luego se crea un Frame para la diapositiva, donde quieras que aparezca,
con el código:
\begin{source}{Titulo}{c5} 
\begin{frame}{Contenido}{} 
\tableofcontents
\end{frame} 
\end{source}
\end{slides}
%
\section{Personalizando el tema {\tt Tesis}}


\subsection{Cambiando los colores }
\begin{slides}{Colores}
Para personalizar el tema {\tt Tesis} una de las opciones es cambiar los colores que trae por defecto, que son los colores representativos de la Universidad del Atlántico.
Se coloca en el preámbulo lo siguiente: 
\begin{source}{Cambiando colores}{c6}
\makeatletter
\definecolor{beamer@barcolor}{cmyk}{0,0.46,0.50,0} % Melon 
\definecolor{beamer@headercolor}{HTML}{B0E0E6} % PowderBlue
% Para cambiar los colores de la barra:
\setbeamercolor{Tesis}{fg=beamer@headercolor!30,bg=beamer@headercolor}
% Para cambiar los colores de la estructura:
\setbeamercolor{structure}{fg=beamer@headercolor}
%  Para cambiar el color del texto de la caja del titulo:
\setbeamercolor{frametitle}{fg=blue}
% Para cambiar el color del texto y del fondo:
\definecolor{azultext}{RGB}{43,93,156}
%  Para los entornos Theorem
\definecolor{colordominante}{cmyk}{0,0.46,0.50,0}
\definecolor{colordominanteF}{RGB}{219,68,14}
\definecolor{colordominanteD}{RGB}{137,46,55}
\makeatother
\end{source}
Recuerde comentar las lineas, de la 119 a la 122.
\end{slides}
\begin{frame}{Colores II}
los cambios que se hicieron en el preámbulo para redefinir los colores nos dan como resultados:
\begin{figure}[H]
\includegraphics[scale=.22]{color-title.png} 
\caption{Diapositiva del t\'itulo}
\label{fig1}
\end{figure}
\end{frame}
\begin{frame}{Colores III}
\begin{figure}[H]
\includegraphics[scale=.7]{ej-color.png} 
\caption{Ejemplo de una diapositiva }
\label{fig1}
\end{figure}
\end{frame}
\begin{frame}{Colores IV}
\begin{figure}[H]
\includegraphics[scale=.2]{tcolorbox.png} 
\caption{Ejemplo de una diapositiva }
\label{fig1}
\end{figure}
Presentaremos ahora una paleta de colores, para que los puedan usar, ya que estan definidos en este tema.
\end{frame}
%%%%%%%%%%%%%%%%%%%%%%%%%%%%%%%%%%%%%%%%%%%%%%%%%%%%%%%%%%%%%%%%%%%%%%%%%%%%
\begin{slides}{Mi paleta de colores }
\begin{table}[H]
\caption{Paleta de colores}
\label{pcolor}
\scalebox{0.5}{
\begin{tabular}{cccccc}
\cdefault[]{naranja1}{rgb}{1,0.5,0} & \cdefault[]{naranja2}{RGB}{255,127,0} & \cdefault[]{naranja3}{cmyk}{0,0.5,1,0} & \cdefault[]{naranja4}{HTML}{FF7F00} & \cdefault[]{rojo}{rgb}{1,0,0}&\cdefault[]{amarillo}{cmyk}{0,0,1,0}\\
\cdefault[]{verde}{rgb}{0,1,0} & \cdefault[]{azul}{rgb}{0,0,1} & \cdefault[]{cian}{cmyk}{1,0,0,0} & \cdefault[]{magenta}{cmyk}{0,1,0,0} &\cdefault[]{negro}{cmyk}{0,0,0,1}&\cdefault[]{blanco}{cmyk}{0,0,0,0}\\% 
\cdefault[]{theblue}{rgb}{0.02,0.04,0.048}&\cdefault[]{thered}{rgb}{0.65,0.04,0.48}&\cdefault[]{thegreen}{rgb}{0.06,0.44,0.08}&\cdefault[]{thegrey}{gray}{0.5}&\cdefault[]{rhodo}{rgb}{0.58,0.63,0.45}&\cdefault[]{wood}{rgb}{0.61,0.51,0.43}\\%
\cdefault[]{theshade}{gray}{0.94}&\cdefault[]{theframe}{gray}{0.75}&\cdefault[]{burl}{rgb}{0.27,0.22,0.20}&\cdefault[]{caper}{rgb}{0.36,0.46,0.23}&\cdefault[]{mesh}{rgb}{0.97,0.93,0.81}&\cdefault[]{warningColor}{named}{red}\\%
\cdefault[]{doc}{RGB}{0,60,110}&\cdefault[]{boxheadcol}{gray}{0.6}&\cdefault[]{boxcol}{gray}{0.9}&\cdefault[named]{PowderBlue}{HTML}{B0E0E6}&\cdefault[named]{MidnightBlue}{HTML}{191970}&\cdefault[]{bl}{rgb}{0,0.2,0.8}\\%
\cdefault[]{shcolor}{HTML}{FDEDD0}&\cdefault[named]{GreenTea}{HTML}{CAE8A2}&\cdefault{MilkTea}{HTML}{C5A16F}&\cdefault{SaddleBrown}{HTML}{8B4513}&\cdefault[]{FrameColor}{rgb}{0.25,0.25,1.0}&\cdefault[]{TitleColor}{rgb}{1.0,1.0,1.0}\\%
\end{tabular} }
\end{table}
\end{slides}
%%%%%%%%%%%%%%%%%%%%%%%%%%%%%%%%%%%%%%%%%%%%%%%%%%%%%%%%%%%%%%%%%%%%%%%%%%%%%%%%
\begin{slides}{Mi paleta de colores I\hspace{-5pt}}
\begin{table}[H]
\caption{Paleta de colores}\label{pcolor}
\scalebox{0.5}{
\begin{tabular}{cccccc} \cdefault[]{TFTitleColor}{HTML}{C5A16F}&\cdefault[]{TFFrameColor}{HTML}{CAE8B3}&\cdefault[]{secnum}{RGB}{13,151,225}&\cdefault[]{ptcbackground}{RGB}{212,237,252}&\cdefault[]{ptctitle}{RGB}{0,177,235}&\cdefault[]{shadecolor}{RGB}{212,237,252}\\%
\cdefault[]{visgreen}{rgb}{0.733,0.776,0}&\cdefault[]{myBGcolor}{HTML}{F6F0D6}&\cdefault{Apricot}{cmyk}{0,0.32,0.52,0}&\cdefault{Melon}{cmyk}{0,0.46,0.50,0}&\cdefault[]{mybrown}{RGB}{128,64,0}&\cdefault[]{lightgrey}{rgb}{0.9,0.9,0.9}\\%
\cdefault[]{darkgreen}{rgb}{0,0.6,0}&\cdefault[]{est1}{RGB}{0,177,235}&\cdefault[]{est2}{RGB}{0,119,158}&\cdefault[]{est3}{RGB}{235,137,0}&\cdefault[]{est4}{RGB}{158,66,0}&\cdefault[]{est5}{RGB}{20,20,20}\\%
\cdefault[]{est6}{RGB}{235,235,235}&\cdefault{RedOrange}{cmyk}{0,0.77,0.87,0}&\cdefault{BlueViolet}{cmyk}{0.86,0.91,0,0.04}&\cdefault{OliveGreen}{cmyk}{0.64,0,0.95,0.40}&\cdefault{Sepia}{cmyk}{0,0.83,1,0.7}&\cdefault{Tan}{cmyk}{0.14,0.42,0.56,0}\\%
\end{tabular}                   }
\end{table}
\end{slides}
%
%%%%%%%%%%%%%%%%%%%%%%%%%%%%%%%%%%%%%%%%%%%%%%%%%%%%%%%%%%%%%%%%%%%%%%%%%%%%%%%%%%%%%%%%%
\makeatletter
\begin{slides}{Mi paleta de colores II\hspace{-4pt}}
\begin{table}[H]
\caption{Paleta de colores}
\label{pcolor}
\scalebox{0.5}{
\begin{tabular}{cccccc}  \cdefault[]{beamer@headercolor}{RGB}{21,46,128}&%naranja 
\cdefault[]{beamer@barcolor}{RGB}{226,107,59}&%azul
\cdefault[]{lightblue}{RGB}{194,193,204}&% light blue 
\cdefault[]{beamer@normaltextcolor}{RGB}{84,97,110}&% gray blue 
\cdefault[]{darkblue}{rgb}{0,0.41,0.54}& % dark blue 
                  %%%-----------otros colores--------- 
\cdefault[]{colortitulo}{RGB}{219,68,14} \\%
     \cdefault[]{colordominante}{RGB}{243,102,25}&    \cdefault[]{colordominanteF}{RGB}{219,68,14}&     \cdefault[]{colordominanteD}{RGB}{137,46,55}&    \cdefault[]{mostaza}{RGB}{231,196,25}&    \cdefault[]{amarilloM}{RGB}{248,199,90}&    \cdefault[]{amarilloD}{RGB}{251,237,121}\\
\cdefault[]{grisamarillo}{RGB}{248,248,245} &     \cdefault[]{azulF}{rgb}{.0,.0,.3}&   \cdefault[]{grisD}{rgb}{.3,.3,.3}&  \cdefault[]{grisF}{rgb}{.6,.6,.6}& \cdefault[]{miverde}{RGB}{44,162,67}&  \cdefault[]{naranjaua}{RGB}{226,107,59}\\
 \cdefault[]{azulua}{RGB}{21,46,128}&\cdefault[]{colorejercicios}{RGB}{99,42,134}& \cdefault[]{colrnodocaja}{RGB}{44,91,144}&\cdefault[]{colrfondocaja}{RGB}{241,241,227}&&\\ 
      \end{tabular}                    } 
\end{table} 
\end{slides}
\makeatother
%%%%%%%%%%%%%%%%%%%%%%%%%%%%%%%%%%%%%%%%%%%%%%%%%%%%%%%%%%%%%%%%%%%%%%%%%%%%%%%%%%%%%%%%%%%%%%
\begin{frame}[fragile]{Entornos o Blocks}
El tema {\tt Tesis } tiene un conjunto amplio de entornos, los cuales los presentaremos todos en las diapositivas que siguentes.

\end{frame}
%%%%%%%%%%%%%%%%%%%%%%%%%%%%%%%%%%%%%%%%%%%%%%%%%%%%%%%%%%%%%%%%%%%%%%%%%%%%%%%%%%%%%%%%%%%%%%%%%%%%%%%%%%%
\begin{frame}[fragile]{Entorno}
\begin{source}{Entorno}{c1}
Formas para utilizar cualquier entorno.
\begin{entorno}
 ...
\end{entorno}

% ?Descripción?
\begin{entorno}[(?Descripción?)]
 ...
\end{entorno}

% ?Descripción? + referencia
\begin{entorno}[(? Descripción ?)][referencia]
 ...
\end{entorno}

% Referencia
\begin{entorno}[][referencia] % [] es obligatorio
 ..
\end{entorno}

\end{source}
\end{frame}
\begin{frame}[fragile]{Ejemplo de una definición}
\begin{source}{Entorno definicion}{}
\begin{definicion}
Definimos la multiplicación entre números naturales como una aplicación
\begin{tabular}{cccc}
 $\times$: & $I\!N\times I\!N$ & $\rightarrow$ & %
$I\!N$%
\tabularnewline
 & $\left(n,m\right)$ & $\mapsto$ & $l=\times\left(n,m\right) = nm$\tabularnewline
\end{tabular}, de modo que $\forall m,n\in I\!N$ se cumple 
\begin{description}
\item [{i)}] $\times\left(1,m\right)=\times\left(m,1\right)=m$
\item [{ii)}] $\times\left(\varphi\left(n\right),m\right)=\times\left(n,m\right)+m.$
\end{description}

\end{definicion}
\end{source}
\scalebox{0.7}{\begin{definicion}
Definimos la multiplicación entre números naturales como una aplicación
\begin{tabular}{cccc}
 $\times$: & $I\!N\times I\!N$ & $\rightarrow$ & %
$I\!N$%
\tabularnewline
 & $\left(n,m\right)$ & $\mapsto$ & $l=\times\left(n,m\right) = nm$\tabularnewline
\end{tabular}, de modo que $\forall m,n\in I\!N$ se cumple 
\begin{description}
\item [{i)}] $\times\left(1,m\right)=\times\left(m,1\right)=m$
\item [{ii)}] $\times\left(\varphi\left(n\right),m\right)=\times\left(n,m\right)+m.$
\end{description}

\end{definicion}}
\end{frame}
%%%%%%%%%%%%%%%%%%%%%%%%%%%%%%%%%%%%%%%%%%%%%%%%%%%%%%%%%%%%%%%%%%%%%%%%%%%%%%
\begin{frame}[fragile]{Ejemplo de un teorema} 
\begin{source}{Entorno definicion}{}
\begin{teorema}
?Ningún? ?número? natural coincide con su sucesor, es decir $\forall n\in\na,\: n\neq\varphi\left(n\right).$
\end{teorema}
\end{source}
\scalebox{0.7}{
\begin{teorema}
Ningún número natural coincide con su sucesor, es decir $\forall n\in I\!N,\: n\neq\varphi\left(n\right).$
\end{teorema}
}
\end{frame}
%%%%%%%%%%%%%%%%%%%%%%%%%%%%%%%%%%%%%%%%%%%%%%%%%%%%%%%%%%%%%%%%%%%%%%%%%%%%%%%%%%
\begin{frame}[fragile]{Ejemplo de una proposición}
\begin{source}{Entorno definicion}{}
\begin{proposicion}
Si $\mathfrak{C}$ es una ?colección? de subconjuntos de $\Omega,$ existe una $\sigma-?álgebra?$ minimal que contiene a $\mathfrak{C}$, esto es, existe una $\sigma-?álgebra?$ $\sigma (\mathfrak{C})$ que contiene a $\mathfrak{C}$  tal que si $\mathfrak{B}$ es otra $\sigma-?álgebra?$ que contiene a $\mathfrak{C}$, $\sigma (\mathfrak{C})\subseteq \mathfrak{B}.$
\end{proposicion}
\end{source}
\scalebox{0.7}{
\begin{proposicion}
Si $\mathfrak{C}$ es una colección de subconjuntos de $\Omega,$ existe una $\sigma-\mbox{\'algebra}$ minimal que contiene a $\mathfrak{C}$, esto es, existe una $\sigma-\mbox{\'algebra}$ $\sigma (\mathfrak{C})$ que contiene a $\mathfrak{C}$  tal que si $\mathfrak{B}$ es otra $\sigma-\mbox{\'algebra}$ que contiene a $\mathfrak{C}$, $\sigma (\mathfrak{C})\subseteq \mathfrak{B}.$
\end{proposicion}
}
\end{frame}
%%%%%%%%%%%%%%%%%%%%%%%%%%%%%%%%%%%%%%%%%%%%%%%%%%%%%%%%%%%%%%%%%%%%%%%%%%%%%%%%%%
\begin{frame}[fragile]{Ejemplo de un lema}
\begin{source}{Entorno definicion}{}
\begin{lema}
Sean $f_1, f_2, f_3. \cdots, f$ funciones medibles
\begin{enumerate}
\itemps Si $f_n\<\geq f$ para todo $n$ y $\int fd\mu > -\infty$, \\
$$\liminf_{n\rightarrow \infty} \int f_nd\mu\geq\int\left( \liminf_{n\rightarrow \infty} f_n \right)d\mu  $$
\itemps Si $f_n\geq f$ para todo $n$ y $\int fd\mu < +\infty$, \\
$$\liminf_{n\rightarrow \infty} \int f_nd\mu\leq\int\left( \liminf_{n\rightarrow \infty} f_n \right)d\mu  $$
\end{enumerate}
\end{lema}
\end{source}
\scalebox{0.7}{
\begin{lema}
Sean $f_1, f_2, f_3. \cdots, f$ funciones medibles
\begin{enumerate}
\itemps Si $f_n\<\geq f$ para todo $n$ y $\int fd\mu > -\infty$, \\
$$\liminf_{n\rightarrow \infty} \int f_nd\mu\geq\int\left( \liminf_{n\rightarrow \infty} f_n \right)d\mu  $$
\itemps Si $f_n\geq f$ para todo $n$ y $\int fd\mu < +\infty$, \\
$$\liminf_{n\rightarrow \infty} \int f_nd\mu\leq\int\left( \liminf_{n\rightarrow \infty} f_n \right)d\mu  $$
\end{enumerate}
\end{lema}
}
\end{frame}
%%%%%%%%%%%%%%%%%%%%%%%%%%%%%%%%%%%%%%%%%%%%%%%%%%%%%%%%%%%%%%%%%%%%%%%%%%%%%%%
%%%%%%%%%%%%%%%%%%%%%%%%%%%%%%%%%%%%%%%%%%%%%%%%%%%%%%%%%%%%%%%%%%%%%%%%%%%%%%%%%%
\begin{frame}[fragile]{Ejemplo de un corolario}
\begin{source}{Entorno definicion}{}
\begin{corolario}\\
 Si \hspace{5pt}$\displaystyle \lim_{x\rightarrow c}f_1(x)=L_1, \displaystyle \lim_{x\rightarrow c}f_2(x)=L_2,\dots, \displaystyle \lim_{x\rightarrow c}f_n(x)=L_n $
 entonces
 \begin{enumerate}
 \item[1.] ${\displaystyle \lim_{x\rightarrow c}}[k_1f_1(x)+k_2f_2(x)+\dots+k_nf_n(x)]=k_1L_1+k_2L_2+\dots+k_nL_n $
 \item[2.]${\displaystyle \lim_{x\rightarrow c}}[f_1(x)f_2(x)\dots f_n(x)]=L_1L_2\dots L_n $
 \end{enumerate}
 \end{corolario}
\end{source}
\scalebox{0.7}{
 \begin{corolario}\\
 Si \hspace{5pt}$\displaystyle \lim_{x\rightarrow c}f_1(x)=L_1, \displaystyle \lim_{x\rightarrow c}f_2(x)=L_2,\dots, \displaystyle \lim_{x\rightarrow c}f_n(x)=L_n $
 entonces
 \begin{enumerate}
 \item[1.] \scalebox{0.9}{${\displaystyle \lim_{x\rightarrow c}}[k_1f_1(x)+k_2f_2(x)+\dots+k_nf_n(x)]=k_1L_1+k_2L_2+\dots+k_nL_n $}
 \item[2.]\scalebox{0.9}{${\displaystyle \lim_{x\rightarrow c}}[f_1(x)f_2(x)\dots f_n(x)]=L_1L_2\dots L_n $}
 \end{enumerate}
 \end{corolario}
}
\end{frame}
%%%%%%%%%%%%%%%%%%%%%%%%%%%%%%%%%%%%%%%%%%%%%%%%%%%%%%%%%%%%%%%%%%%%%%%%%%%%%%%%%%%%%
\begin{frame}[fragile]{Presentemos un ejemplo}
\begin{source}{Entorno definicion}{}
\begin{ejemplo}
Demuestre que si $p(x)=a_0+a_1x+\dots + a_nx^n$ entonces
 $$\lim_{x\rightarrow c}p(x)=p(c)$$
 \end{ejemplo}
\end{source}
\scalebox{0.7}{
\begin{ejemplo}
Demuestre que si $p(x)=a_0+a_1x+\dots + a_nx^n$ entonces
 $$\lim_{x\rightarrow c}p(x)=p(c)$$
 \end{ejemplo}
}
\end{frame}
%%%%%%%%%%%%%%%%%%%%%%%%%%%%%%%%%%%%%%%%%%%%%%%%%%%%%%%%%%%%%%%%%%%%%%%%%%%%%%%%%%%%%%%%
\begin{frame}[fragile]{Presentemos una scaja}
\begin{source}{Entorno definicion}{}
\begin{scaja}
Sea $f:\, A\overset{f}{\rightarrow}B$ una ?función? de $A$
en $B$, llamaremos dominio de la ?función? $f$, al conjunto de todas
las primeras componentes, el cual denotaremos por $D_{f}$ , es decir:
\[
D_{f}=\left\{ x\in A:\:\mbox{tal que existe un }y\in B\wedge\left(x,y\right)\in f\right\} \subseteq A.
\]
 \end{scaja}
\end{source}
\scalebox{0.7}{
\begin{scaja}
Sea $f:\, A\overset{f}{\rightarrow}B$ una función de $A$
en $B$, llamaremos dominio de la función $f$, al conjunto de todas
las primeras componentes, el cual denotaremos por $D_{f}$ , es decir:
\[
D_{f}=\left\{ x\in A:\:\mbox{tal que existe un }y\in B\wedge\left(x,y\right)\in f\right\} \subseteq A.
\]
 \end{scaja}
}
\end{frame}
%%%%%%%%%%%%%%%%%%%%%%%%%%%%%%%%%%%%%%%%%%%%%%%%%%%%%%%%%%%%%%%%%%%%%%%%%%%%%%%%%%%%%%%%
\begin{frame}[fragile]{Presentemos el entorno vocabulario}
\begin{source}{Entorno definicion}{}
\begin{vocabulario}
Sea $f:\, A\overset{f}{\rightarrow}B$ una ?función? de $A$
en $B$, llamaremos dominio de la ?función? $f$, al conjunto de todas
las primeras componentes, el cual denotaremos por $D_{f}$ , es decir:
\[
D_{f}=\left\{ x\in A:\:\mbox{tal que existe un }y\in B\wedge\left(x,y\right)\in f\right\} \subseteq A.
\]
 \end{vocabulario}
\end{source}
\scalebox{0.7}{
\begin{vocabulario}
Sea $f:\, A\overset{f}{\rightarrow}B$ una función de $A$
en $B$, llamaremos dominio de la función $f$, al conjunto de todas
las primeras componentes, el cual denotaremos por $D_{f}$ , es decir:
\[
D_{f}=\left\{ x\in A:\:\mbox{tal que existe un }y\in B\wedge\left(x,y\right)\in f\right\} \subseteq A.
\]
 \end{vocabulario}
}
\end{frame}
%%%%%%%%%%%%%%%%%%%%%%%%%%%%%%%%%%%%%%%%%%%%%%%%%%%%%%%%%%%%%%%%%%%%%%%%%%%%%%%%%%%%%%%%
\begin{frame}[fragile]{Presentemos el entorno cajaejercicios}
\begin{source}{Entorno definicion}{}
\begin{vocabulario}
\begin{cajaejercicios}
Sí $p:\:4<7$ y $q:\ 6\,\mbox{ es ?número? par}$.
Calcular el valor de verdad de $p\wedge q.$ 
 \end{cajaejercicios}
\end{source}
\scalebox{0.7}{
\begin{cajaejercicios}
Sí $p:\:4<7$ y $q:\ 6\,\mbox{ es número par}$.
Calcular el valor de verdad de $p\wedge q.$ 
 \end{cajaejercicios}
}
\end{frame}
%%%%%%%%%%%%%%%%%%%%%%%%%%%%%%%%%%%%%%%%%%%%%%%%%%%%%%%%%%%%%%%%%%%%%%%%%%%%%%%%%%%%%%%%
\begin{frame}[fragile]{Presentemos el entorno proof}
\begin{source}{Entorno definicion}{}
\begin{proof}
Para probar que las proposiciones $(p\rightarrow q)$ y
$(\sim q\rightarrow\sim p)$ son ?lógicamente? equivalentes debemos
probar que $(p\rightarrow q)$ $\leftrightarrow$ $(\sim q\rightarrow\sim p)$
es una ?tautología?. 
\end{proof}
\end{source}
\scalebox{0.7}{
\begin{proof}
Para probar que las proposiciones $(p\rightarrow q)$ y
$(\sim q\rightarrow\sim p)$ son lógicamente equivalentes debemos
probar que $(p\rightarrow q)$ $\leftrightarrow$ $(\sim q\rightarrow\sim p)$
es una tautología. 
\end{proof}
}
\end{frame}
%%%%%%%%%%%%%%%%%%%%%%%%%%%%%%%%%%%%%%%%%%%%%%%%%%%%%%%%%%%%%%%%%%%%%%%%%%%%%%%%%%%%%%%%
\begin{frame}[fragile]{Presentemos el entorno prueba}
\begin{source}{Entorno definicion}{}
\begin{prueba}
Para probar que las proposiciones $(p\rightarrow q)$ y
$(\sim q\rightarrow\sim p)$ son ?lógicamente? equivalentes debemos
probar que $(p\rightarrow q)$ $\leftrightarrow$ $(\sim q\rightarrow\sim p)$
es una ?tautología?. 
\end{prueba}
\end{source}
%\scalebox{0.7}{
\begin{prueba}
Para probar que las proposiciones $(p\rightarrow q)$ y
$(\sim q\rightarrow\sim p)$ son lógicamente equivalentes debemos
probar que $(p\rightarrow q)$ $\leftrightarrow$ $(\sim q\rightarrow\sim p)$
es una tautología. 
\end{prueba}
%}
\end{frame}
\begin{frame}[fragile]{Block sin título}
\begin{source}{Entorno definicion}{}
\begin{block}<1->{}
\begin{itemize}
\item Bloque sin ?título?.
\item Mostrado en todas las diapositivas.
\end{itemize}
\end{block}
\end{source}
%\scalebox{0.7}{
\begin{block}<1->{}
\begin{itemize}
\item Bloque sin título.
\item Mostrado en todas las diapositivas.
\end{itemize}
\end{block}
%}
\end{frame}
%%%%%%%%%%%%%%%%%%%%%%%%%%%%%%%%%%%%%%%%%%%%%%%%%%%%%%%%%%%%%%%%%%%%%%%%%%%%%%%%%%%%%%%
\begin{frame}[fragile]{Block de ejemplo con título}
\begin{source}{Entorno definicion}{}
\begin{exampleblock}<2->{?Título? de ?algún? Bloque de ejemplo}
\begin{itemize}
\item $e^{i\pi}=-1$.
\item $e^{i\pi/2}=i$.
\end{itemize}
\end{exampleblock}
\end{source}
%\scalebox{0.7}{
\begin{exampleblock}<2->{Título de algún Bloque de ejemplo}

\begin{itemize}
\item $e^{i\pi}=-1$.
\item $e^{i\pi/2}=i$.
\end{itemize}
\end{exampleblock}
%}
\end{frame}
%%%%%%%%%%%%%%%%%%%%%%%%%%%%%%%%%%%%%%%%%%%%%%%%%%%%%%%%%%%%%%%%%%%%%%%%%%%%%%%%%%%%%%%
\begin{frame}[fragile]{Block example}
\begin{source}{Entorno definicion}{}
\begin{example}
\begin{itemize}
\item $e^{i\pi}=-1$.
\item $e^{i\pi/2}=i$.
\end{itemize}
\end{example}
\end{source}
%\scalebox{0.7}{
\begin{example}
\begin{itemize}
\item $e^{i\pi}=-1$.
\item $e^{i\pi/2}=i$.
\end{itemize}
\end{example}
%}
\end{frame}
%%%%%%%%%%%%%%%%%%%%%%%%%%%%%%%%%%%%%%%%%%%%%%%%%%%%%%%%%%%%%%%%%%%%%%%%%%%%%%%%%%%%%%%
%%%%%%%%%%%%%%%%%%%%%%%%%%%%%%%%%%%%%%%%%%%%%%%%%%%%%%%%%%%%%%%%%%%%%%%%%%%%%%%%%%%%%%%
\begin{frame}[fragile]{Block alertblock}
\begin{source}{Entorno definicion}{}
\begin{alertblock}{Un block para alertas}
\begin{itemize}
\item $e^{i\pi}=-1$.
\item $e^{i\pi/2}=i$.
\end{itemize}
\end{alertblock}
\end{source}
%\scalebox{0.7}{
\begin{alertblock}{Un block para alertas}
\begin{itemize}
\item $e^{i\pi}=-1$.
\item $e^{i\pi/2}=i$.
\end{itemize}
\end{alertblock}
%}
\end{frame}
%%%%%%%%%%%%%%%%%%%%%%%%%%%%%%%%%%%%%%%%%%%%%%%%%%%%%%%%%%%%%%%%%%%%%%%%%%%%%%%%%%%%%%%
\begin{frame}[fragile]{Block theorem}
\begin{source}{Entorno definicion}{}
\begin{theorem}
?Ningún? ?número? natural coincide con su sucesor, es decir $\forall n\in I\!N,\: n\neq\varphi\left(n\right).$
\end{theorem}
\end{source}
%\scalebox{0.7}{
\begin{theorem}
Ningún número natural coincide con su sucesor, es decir $\forall n\in I\!N,\: n\neq\varphi\left(n\right).$
\end{theorem}

%}
\end{frame}
%%%%%%%%%%%%%%%%%%%%%%%%%%%%%%%%%%%%%%%%%%%%%%%%%%%%%%%%%%%%%%%%%%%%%%%%%%%%%%%%%%%%%%%
%%%%%%%%%%%%%%%%%%%%%%%%%%%%%%%%%%%%%%%%%%%%%%%%%%%%%%%%%%%%%%%%%%%%%%%%%%%%%%%%%%%%%%%
\begin{frame}[fragile]{Block corollary}
\begin{source}{Entorno definicion}{}
\begin{corollary}
 Si \hspace{5pt}$\displaystyle \lim_{x\rightarrow c}f_1(x)=L_1, \displaystyle \lim_{x\rightarrow c}f_2(x)=L_2,\dots, \displaystyle \lim_{x\rightarrow c}f_n(x)=L_n $
 entonces
\end{corollary}
\end{source}
%\scalebox{0.7}{
\begin{corollary}
 Si \hspace{5pt}$\displaystyle \lim_{x\rightarrow c}f_1(x)=L_1, \displaystyle \lim_{x\rightarrow c}f_2(x)=L_2,\dots, \displaystyle \lim_{x\rightarrow c}f_n(x)=L_n $
 entonces
\end{corollary}
%}
\end{frame}
%%%%%%%%%%%%%%%%%%%%%%%%%%%%%%%%%%%%%%%%%%%%%%%%%%%%%%%%%%%%%%%%%%%%%%%%%%%%%%%%%%%%%%%
%%%%%%%%%%%%%%%%%%%%%%%%%%%%%%%%%%%%%%%%%%%%%%%%%%%%%%%%%%%%%%%%%%%%%%%%%%%%%%%%%%%%%%%
\begin{frame}[fragile]{Block example}
\begin{source}{Entorno definicion}{}
\begin{example}
 Si \hspace{5pt}$\displaystyle \lim_{x\rightarrow c}f_1(x)=L_1, \displaystyle \lim_{x\rightarrow c}f_2(x)=L_2,\dots, \displaystyle \lim_{x\rightarrow c}f_n(x)=L_n $
 entonces
\end{example}
\end{source}
%\scalebox{0.7}{
\begin{example}
 Si \hspace{5pt}$\displaystyle \lim_{x\rightarrow c}f_1(x)=L_1, \displaystyle \lim_{x\rightarrow c}f_2(x)=L_2,\dots, \displaystyle \lim_{x\rightarrow c}f_n(x)=L_n $
 entonces
\end{example}
%}
\end{frame}

%%%%%%%%%%%%%%%%%%%%%%%%%%%%%%%%%%%%%%%%%%%%%%%%%%%%%%%%%%%%%%%%%%%%%%%%%%%%%%%%%%%%%%%
\begin{frame}[fragile]{Block examples}
\begin{source}{Entorno definicion}{}
\begin{examples}
 Si \hspace{5pt}$\displaystyle \lim_{x\rightarrow c}f_1(x)=L_1, \displaystyle \lim_{x\rightarrow c}f_2(x)=L_2,\dots, \displaystyle \lim_{x\rightarrow c}f_n(x)=L_n $
 entonces
\end{examples}
\end{source}
%\scalebox{0.7}{
\begin{examples}
 Si \hspace{5pt}$\displaystyle \lim_{x\rightarrow c}f_1(x)=L_1, \displaystyle \lim_{x\rightarrow c}f_2(x)=L_2,\dots, \displaystyle \lim_{x\rightarrow c}f_n(x)=L_n $
 entonces
\end{examples}
%}
\end{frame}
%%%%%%%%%%%%%%%%%%%%%%%%%%%%%%%%%%%%%%%%%%%%%%%%%%%%%%%%%%%%%%%%%%%%%%%%%%%%%%%%%%%%%%%
\begin{frame}[fragile]{Block definition}
\begin{source}{Entorno definicion}{}
\begin{definition}
 Si \hspace{5pt}$\displaystyle \lim_{x\rightarrow c}f_1(x)=L_1, \displaystyle \lim_{x\rightarrow c}f_2(x)=L_2,\dots, \displaystyle \lim_{x\rightarrow c}f_n(x)=L_n $
 entonces
\end{definition}
\end{source}
%\scalebox{0.7}{
\begin{definition}
 Si \hspace{5pt}$\displaystyle \lim_{x\rightarrow c}f_1(x)=L_1, \displaystyle \lim_{x\rightarrow c}f_2(x)=L_2,\dots, \displaystyle \lim_{x\rightarrow c}f_n(x)=L_n $
 entonces
\end{definition}
%}
\end{frame}
%%%%%%%%%%%%%%%%%%%%%%%%%%%%%%%%%%%%%%%%%%%%%%%%%%%%%%%%%%%%%%%%%%%%%%%%%%%%%%%%%%%%%%%
\begin{frame}[fragile]{Block fact}
\begin{source}{Entorno definicion}{}
\begin{fact}
 Si \hspace{5pt}$\displaystyle \lim_{x\rightarrow c}f_1(x)=L_1, \displaystyle \lim_{x\rightarrow c}f_2(x)=L_2,\dots, \displaystyle \lim_{x\rightarrow c}f_n(x)=L_n $
 entonces
\end{fact}
\end{source}
%\scalebox{0.7}{
\begin{fact}
 Si \hspace{5pt}$\displaystyle \lim_{x\rightarrow c}f_1(x)=L_1, \displaystyle \lim_{x\rightarrow c}f_2(x)=L_2,\dots, \displaystyle \lim_{x\rightarrow c}f_n(x)=L_n $
 entonces
\end{fact}
%}
\end{frame}
%%%%%%%%%%%%%%%%%%%%%%%%%%%%%%%%%%%%%%%%%%%%%%%%%%%%%%%%%%%%%%%%%%%%%%%%%%%%%%%%%%%%%%%
\begin{frame}[fragile]{Block definitions}
\begin{source}{Entorno definicion}{}
\begin{definitions}
 Si \hspace{5pt}$\displaystyle \lim_{x\rightarrow c}f_1(x)=L_1, \displaystyle \lim_{x\rightarrow c}f_2(x)=L_2,\dots, \displaystyle \lim_{x\rightarrow c}f_n(x)=L_n $
 entonces
\end{definitions}
\end{source}
%\scalebox{0.7}{
\begin{definitions}
 Si \hspace{5pt}$\displaystyle \lim_{x\rightarrow c}f_1(x)=L_1, \displaystyle \lim_{x\rightarrow c}f_2(x)=L_2,\dots, \displaystyle \lim_{x\rightarrow c}f_n(x)=L_n $
 entonces
\end{definitions}
%}
\end{frame}
%%%%%%%%%%%%%%%%%%%%%%%%%%%%%%%%%%%%%%%%%%%%%%%%%%%%%%%%%%%%%%%%%%%%%%%%%%%%%%%%%%%%%%%
\begin{frame}[fragile]{Block definitionT}
\begin{source}{Entorno definicion}{}
\begin{definitionT}
 Si \hspace{5pt}$\displaystyle \lim_{x\rightarrow c}f_1(x)=L_1, \displaystyle \lim_{x\rightarrow c}f_2(x)=L_2,\dots, \displaystyle \lim_{x\rightarrow c}f_n(x)=L_n $
 entonces
\end{definitionT}
\end{source}
%\scalebox{0.7}{
\begin{definitionT}
 Si \hspace{5pt}$\displaystyle \lim_{x\rightarrow c}f_1(x)=L_1, \displaystyle \lim_{x\rightarrow c}f_2(x)=L_2,\dots, \displaystyle \lim_{x\rightarrow c}f_n(x)=L_n $
 entonces
\end{definitionT}
%}
\end{frame}
%%%%%%%%%%%%%%%%%%%%%%%%%%%%%%%%%%%%%%%%%%%%%%%%%%%%%%%%%%%%%%%%%%%%%%%%%%%%%%%%%%%%%%%%%%%%%%%%%%
\begin{frame}[fragile]{Block corollaryT}
\begin{source}{Entorno definicion}{}
\begin{corollaryT}
 Si \hspace{5pt}$\displaystyle \lim_{x\rightarrow c}f_1(x)=L_1, \displaystyle \lim_{x\rightarrow c}f_2(x)=L_2,\dots, \displaystyle \lim_{x\rightarrow c}f_n(x)=L_n $
 entonces
\end{corollaryT}
\end{source}
%\scalebox{0.7}{
\begin{corollaryT}
 Si \hspace{5pt}$\displaystyle \lim_{x\rightarrow c}f_1(x)=L_1, \displaystyle \lim_{x\rightarrow c}f_2(x)=L_2,\dots, \displaystyle \lim_{x\rightarrow c}f_n(x)=L_n $
 entonces
\end{corollaryT}
%}
\end{frame}
%%%%%%%%%%%%%%%%%%%%%%%%%%%%%%%%%%%%%%%%%%%%%%%%%%%%%%%%%%%%%%%%%%%%%%%%%%%%%%%%%%%%%%%%%%%%%%%%%%
\begin{frame}[fragile]{Block ejerciciosT}
\begin{source}{Entorno definicion}{}
\begin{ejerciciosT}
 Si \hspace{5pt}$\displaystyle \lim_{x\rightarrow c}f_1(x)=L_1, \displaystyle \lim_{x\rightarrow c}f_2(x)=L_2,\dots, \displaystyle \lim_{x\rightarrow c}f_n(x)=L_n $
 entonces
\end{ejerciciosT}
\end{source}
%\scalebox{0.7}{
\begin{ejerciciosT}
 Si \hspace{5pt}$\displaystyle \lim_{x\rightarrow c}f_1(x)=L_1, \displaystyle \lim_{x\rightarrow c}f_2(x)=L_2,\dots, \displaystyle \lim_{x\rightarrow c}f_n(x)=L_n $
 entonces
\end{ejerciciosT}
%}
\end{frame}
%%%%%%%%%%%%%%%%%%%%%%%%%%%%%%%%%%%%%%%%%%%%%%%%%%%%%%%%%%%%%%%%%%%%%%%%%%%%%%%%%%%%%%%%%%%%%%%%%%
\begin{frame}[fragile]{Block theoremeT}
\begin{source}{Entorno definicion}{}
\begin{theoremeT}
 Si \hspace{5pt}$\displaystyle \lim_{x\rightarrow c}f_1(x)=L_1, \displaystyle \lim_{x\rightarrow c}f_2(x)=L_2,\dots, \displaystyle \lim_{x\rightarrow c}f_n(x)=L_n $
 entonces
\end{theoremeT}
\end{source}
%\scalebox{0.7}{
\begin{theoremeT}
 Si \hspace{5pt}$\displaystyle \lim_{x\rightarrow c}f_1(x)=L_1, \displaystyle \lim_{x\rightarrow c}f_2(x)=L_2,\dots, \displaystyle \lim_{x\rightarrow c}f_n(x)=L_n $
 entonces
\end{theoremeT}
%}
\end{frame}
%%%%%%%%%%%%%%%%%%%%%%%%%%%%%%%%%%%%%%%%%%%%%%%%%%%%%%%%%%%%%%%%%%%%%%%%%%%%%%%%%%%%%%%%%%%%%%%%%%
\begin{frame}[fragile]{Block Problema}
\begin{source}{Entorno definicion}{}
\begin{problema}
 Si \hspace{5pt}$\displaystyle \lim_{x\rightarrow c}f_1(x)=L_1, \displaystyle \lim_{x\rightarrow c}f_2(x)=L_2,\dots, \displaystyle \lim_{x\rightarrow c}f_n(x)=L_n $
 entonces
\end{problema}
\end{source}
%\scalebox{0.7}{
\begin{problema}
 Si \hspace{5pt}$\displaystyle \lim_{x\rightarrow c}f_1(x)=L_1, \displaystyle \lim_{x\rightarrow c}f_2(x)=L_2,\dots, \displaystyle \lim_{x\rightarrow c}f_n(x)=L_n $
 entonces
\end{problema}
%}
\end{frame}
%%%%%%%%%%%%%%%%%%%%%%%%%%%%%%%%%%%%%%%%%%%%%%%%%%%%%%%%%%%%%%%%%%%%%%%%%%%%%%%%%%%%%%%%%%%%%%%%%%
\begin{frame}[fragile]{Block exerciseT}
\begin{source}{Entorno definicion}{}
\begin{exerciseT}
 Si \hspace{5pt}$\displaystyle \lim_{x\rightarrow c}f_1(x)=L_1, \displaystyle \lim_{x\rightarrow c}f_2(x)=L_2,\dots, \displaystyle \lim_{x\rightarrow c}f_n(x)=L_n $
 entonces
\end{exerciseT}
\end{source}
%\scalebox{0.7}{
\begin{exerciseT}
 Si \hspace{5pt}$\displaystyle \lim_{x\rightarrow c}f_1(x)=L_1, \displaystyle \lim_{x\rightarrow c}f_2(x)=L_2,\dots, \displaystyle \lim_{x\rightarrow c}f_n(x)=L_n $
 entonces
\end{exerciseT}
%}
\end{frame}
%%%%%%%%%%%%%%%%%%%%%%%%%%%%%%%%%%%%%%%%%%%%%%%%%%%%%%%%%%%%%%%%%%%%%%%%%%%%%%%%%%%%%%%%%%%%%%%%%%
\begin{frame}[fragile]{Block ejercicio}
\begin{source}{Entorno definicion}{}
\begin{ejercicio}
 Si \hspace{5pt}$\displaystyle \lim_{x\rightarrow c}f_1(x)=L_1, \displaystyle \lim_{x\rightarrow c}f_2(x)=L_2,\dots, \displaystyle \lim_{x\rightarrow c}f_n(x)=L_n $
 entonces
\end{ejercicio}
\end{source}
%\scalebox{0.7}{
\begin{ejercicio}
 Si \hspace{5pt}$\displaystyle \lim_{x\rightarrow c}f_1(x)=L_1, \displaystyle \lim_{x\rightarrow c}f_2(x)=L_2,\dots, \displaystyle \lim_{x\rightarrow c}f_n(x)=L_n $
 entonces
\end{ejercicio}
%}
\end{frame}
%%%%%%%%%%%%%%%%%%%%%%%%%%%%%%%%%%%%%%%%%%%%%%%%%%%%%%%%%%%%%%%%%%%%%%%%%%%%%%%%%%%%%%%%%%%%%%%%%%
\begin{frame}[fragile]{Block nota}
\begin{source}{Entorno definicion}{}
\begin{nota}
 Si \hspace{5pt}$\displaystyle \lim_{x\rightarrow c}f_1(x)=L_1, \displaystyle \lim_{x\rightarrow c}f_2(x)=L_2,\dots, \displaystyle \lim_{x\rightarrow c}f_n(x)=L_n $
 entonces
\end{nota}
\end{source}
%\scalebox{0.7}{
\begin{nota}
 Si \hspace{5pt}$\displaystyle \lim_{x\rightarrow c}f_1(x)=L_1, \displaystyle \lim_{x\rightarrow c}f_2(x)=L_2,\dots, \displaystyle \lim_{x\rightarrow c}f_n(x)=L_n $
 entonces
\end{nota}
%}
\end{frame}
%%%%%%%%%%%%%%%%%%%%%%%%%%%%%%%%%%%%%%%%%%%%%%%%%%%%%%%%%%%%%%%%%%%%%%%%%%%%%%%%%%%%%%%%%%%%%%%%%%
\begin{frame}[fragile]{Block exer}
\begin{source}{Entorno definicion}{}
\begin{exer}
 Si \hspace{5pt}$\displaystyle \lim_{x\rightarrow c}f_1(x)=L_1, \displaystyle \lim_{x\rightarrow c}f_2(x)=L_2,\dots, \displaystyle \lim_{x\rightarrow c}f_n(x)=L_n $
 entonces
\end{exer}
\end{source}
%\scalebox{0.7}{
\begin{exer}
 Si \hspace{5pt}$\displaystyle \lim_{x\rightarrow c}f_1(x)=L_1, \displaystyle \lim_{x\rightarrow c}f_2(x)=L_2,\dots, \displaystyle \lim_{x\rightarrow c}f_n(x)=L_n $
 entonces
\end{exer}
%}
\end{frame}
%%%%%%%%%%%%%%%%%%%%%%%%%%%%%%%%%%%%%%%%%%%%%%%%%%%%%%%%%%%%%%%%%%%%%%%%%%%%%%%%%%%%%%%%%%%%%%%%%%
\begin{frame}[fragile]{Block ejer}
\begin{source}{Entorno definicion}{}
\begin{ejer}
 Si \hspace{5pt}$\displaystyle \lim_{x\rightarrow c}f_1(x)=L_1, \displaystyle \lim_{x\rightarrow c}f_2(x)=L_2,\dots, \displaystyle \lim_{x\rightarrow c}f_n(x)=L_n $
 entonces
\end{ejer}
\end{source}
%\scalebox{0.7}{
\begin{ejer}
 Si \hspace{5pt}$\displaystyle \lim_{x\rightarrow c}f_1(x)=L_1, \displaystyle \lim_{x\rightarrow c}f_2(x)=L_2,\dots, \displaystyle \lim_{x\rightarrow c}f_n(x)=L_n $
 entonces
\end{ejer}
%}
\end{frame}
%%%%%%%%%%%%%%%%%%%%%%%%%%%%%%%%%%%%%%%%%%%%%%%%%%%%%%%%%%%%%%%%%%%%%%%%%%%%%%%%%%%%%%%%%%%%%%%%%%
\begin{frame}[fragile]{Block solu}
\begin{source}{Entorno definicion}{}
\begin{solu}
 Si \hspace{5pt}$\displaystyle \lim_{x\rightarrow c}f_1(x)=L_1, \displaystyle \lim_{x\rightarrow c}f_2(x)=L_2,\dots, \displaystyle \lim_{x\rightarrow c}f_n(x)=L_n $
 entonces
\end{solu}
\end{source}
%\scalebox{0.7}{
\begin{solu}
 Si \hspace{5pt}$\displaystyle \lim_{x\rightarrow c}f_1(x)=L_1, \displaystyle \lim_{x\rightarrow c}f_2(x)=L_2,\dots, \displaystyle \lim_{x\rightarrow c}f_n(x)=L_n $
 entonces
\end{solu}
%}
\end{frame}
%%%%%%%%%%%%%%%%%%%%%%%%%%%%%%%%%%%%%%%%%%%%%%%%%%%%%%%%%%%%%%%%%%%%%%%%%%%%%%%%%%%%%%%%%%%%%%%%%%
\begin{frame}[fragile]{Block ejercicios}
\begin{source}{Entorno definicion}{}
\begin{definitions}
 Si \hspace{5pt}$\displaystyle \lim_{x\rightarrow c}f_1(x)=L_1, \displaystyle \lim_{x\rightarrow c}f_2(x)=L_2,\dots, \displaystyle \lim_{x\rightarrow c}f_n(x)=L_n $
 entonces
\end{definitions}
\end{source}
%\scalebox{0.7}{
\begin{ejercicios}
 Si \hspace{5pt}$\displaystyle \lim_{x\rightarrow c}f_1(x)=L_1, \displaystyle \lim_{x\rightarrow c}f_2(x)=L_2,\dots, \displaystyle \lim_{x\rightarrow c}f_n(x)=L_n $
 entonces
\end{ejercicios}
%}
\end{frame}
%%%%%%%%%%%%%%%%%%%%%%%%%%%%%%%%%%%%%%%%%%%%%%%%%%%%%%%%%%%%%%%%%%%%%%%%%%%%%%%%%%%%%%%%%%%%%%%%%%
%%%%%%%%%%%%%%%%%%%%%%%%%%%%%%%%%%%%%%%%%%%%%%%%%%%%%%%%%%%%%%%%%%%%%%%%%%%%%%%%%%%%%%%%%%%%%%%%%%
\begin{frame}[fragile]{Block caja}
\begin{source}{Entorno definicion}{}
\begin{caja}
 Si \hspace{5pt}$\displaystyle \lim_{x\rightarrow c}f_1(x)=L_1, \displaystyle \lim_{x\rightarrow c}f_2(x)=L_2,\dots, \displaystyle \lim_{x\rightarrow c}f_n(x)=L_n $
 entonces
\end{caja}
\end{source}
%\scalebox{0.7}{
\begin{caja}
 Si \hspace{5pt}$\displaystyle \lim_{x\rightarrow c}f_1(x)=L_1, \displaystyle \lim_{x\rightarrow c}f_2(x)=L_2,\dots, \displaystyle \lim_{x\rightarrow c}f_n(x)=L_n $
 entonces
\end{caja}
%}
\end{frame}
%%%%%%%%%%%%%%%%%%%%%%%%%%%%%%%%%%%%%%%%%%%%%%%%%%%%%%%%%%%%%%%%%%%%%%%%%%%%%%%%%%%%%%%%%%%%%%%%%%

\end{document}
