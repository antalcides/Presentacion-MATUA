\batchmode
\makeatletter
\def\input@path{{style/}{sections/}{pdf/}{logos/}}
\makeatother
\documentclass[]{beamer}


%-------------------------------------------------------
% Inclusión de paquetes
%-------------------------------------------------------
%\setbeamertemplate{frametitle continuation}{\frametitle{\color{white}Title}}
\usepackage{lipsum-es}
%\usepackage[latin9]{inputenc}
%\usepackage[ansinew]{inputenc}
%--------------------------------
%
\usepackage{ifthen} 
\usepackage[T1]{fontenc}
\usepackage[mathletters]{ucs}
\usepackage[utf8x]{inputenc}
\usepackage[english,spanish,es-tabla]{babel}
\uselanguage{spanish}
\languagepath{spanish}
\usepackage{environ}
\usepackage{lmodern}
\usefonttheme[onlymath]{serif}
\usefonttheme{professionalfonts}
%\usepackage[spanish]{babel}
\usepackage{times}



%---------------------------------
\usepackage{amssymb,amsfonts,latexsym,cancel,stmaryrd}%
\usepackage[ruled,,vlined,lined,linesnumbered]{algorithm2e}
\usepackage{framed}
\usepackage{mathptmx}
\usepackage{helvet}
%\linespread{1.05}
\usepackage[full]{textcomp}                        % Caracteres especiales como ' (recto)
\usepackage{amsmath}
\makeatletter
\let\@tmp\@xfloat     
\usepackage{fixltx2e}
\let\@xfloat\@tmp                    
\makeatother
\makeatletter
\let\th@plain\relax
\makeatother
\theoremstyle{plain}
\let\openbox\relax
%\usepackage{ntheorem}
\let\proofname\relax
\let\proof\relax
\let\endproof\relax
\usepackage{amsthm}
%\usepackage{ddot}
%\usepackage{a4wide}
\usepackage{amsfonts}
\usepackage{epsfig}
%\usepackage{amscd}
\usepackage{longtable}
%\usepackage{latexcad}
\usepackage{fancybox}
\graphicspath{{ps/}{logos/}{figuras/}{sections/Figures/}}
\listfiles
% COMANDOS PERSONALES   ----------------------------------------------
\newcommand{\R}{\mathbb{R}}
\newcommand{\Z}{\mathbb{Z}}
\newcommand{\Q}{\mathbb{Q}}
\newcommand{\N}{\mathbb{N}}
\newcommand{\I}{\mathbb{I}}
\newcommand{\raya}{\rule{2cm}{0.01cm}\\}
\newcommand{\ds}{\displaystyle}
\newcommand{\sen}{\mathop{\rm sen}\nolimits}
\newcommand{\senh}{\mathop{\rm senh}\nolimits}
\newcommand{\arcsen}{\mathop{\rm arcsen}\nolimits}
\newcommand{\arcsec}{\mathop{\rm arcsec}\nolimits}
%\def\sen{\mathop{\mbox{\normalfont sen}}\nolimits}
\def\max{\mathop{\mbox{\normalfont m\’ax}}}
\newcommand{\bc}{\begin{center}}
\newcommand{\ec}{\end{center}}
\newcommand{\be}{\begin{enumerate}}
\newcommand{\ee}{\end{enumerate}}
%----------- cargando el Tema Tesis
\DeclareGraphicsExtensions{.eps,.jpg,.png, .bmp}


\usetheme[]{Tesis}%%% opción para el tema outher.
%%%%%%%%%%%%%%%%%%%%%%%% configurando colores y block %%%%%%%%%%%%%%%%%%%%%%%%%%%%%%%%%%%%%
%%%%%%%%%%%%%%%%%%%%%%%%%%%%%%%%%%%%%%%%%%%%%%%%%%%%%%%%%%%%%%%%%%%%%%%%%%%%%%%%%%%%%%%%%
\setbeamercovered{transparent}
% or whatever (possibly just delete it)
%%%%%% ncambiando color de blocks%%%%%
%\setbeamercolor{itemize item}{fg=red} % all frames will have red bullets no funciona en esta version
\setbeamertemplate{blocks}[rounded][shadow=true]
%\addtobeamertemplate{block begin}{\pgfsetfillopacity{0.8}}{\pgfsetfillopacity{1}}
\setbeamercolor{structure}{fg=gray!80}
\setbeamercolor*{block title example}{fg=colordominante,
bg= azulua!70}
\setbeamercolor*{block title alerted}{fg=azulua,
bg= naranjaua!70}
\setbeamercolor*{block body example}{fg= black,
bg= azulua!10}
 \addtobeamertemplate{block begin}{\pgfsetfillopacity{0.5}}{\pgfsetfillopacity{1}}
 \addtobeamertemplate{block alerted begin}{\pgfsetfillopacity{0.5}}{\pgfsetfillopacity{1}}
 \addtobeamertemplate{block example begin}{\pgfsetfillopacity{0.5}}{\pgfsetfillopacity{1}}
 %%%%%%%%%
 \setbeamercolor*{item}{fg=azulua}
\setbeamercolor{title}{bg=colordominante}
\setbeamercolor{frametitle}{bg=colordominante}
\setbeamercolor{section in toc}{fg=colordominante}
\setbeamercolor{subsection in toc}{fg=red}
%%%%%%%%%%%%%%%%%%%%%%%%%%%%%%%%%%%%%%%%%%%%%%%%%%%%%%%%%%%%%%%%%%%%%%%%%%%%%%%%%%%%%%
%%%%%%%%%%%%%%%%%%%%%%%%%%%%%%%%%%%%%%%%%%%%%%%%%%%%%%%%%%%%%%%%%%%%%%%%%%%%%%%%%%%%%%%%%
%%    progressstyle=fixedCircCnt,   % fixedCircCnt, movingCircCnt (Movimiento por defecto)
%%  para los colores de las block de  teoremas
%\definecolor{colordominante}{cmyk}{0,0.46,0.50,0} % melon
%  
%% Si desea cambiar los colores de los diversos elementos en el tema, edite y descomente las siguientes líneas
%\makeatletter
%% Para cambiar los colores de la barra:
%\setbeamercolor{Tesis}{fg=beamer@headercolor!30,bg=beamer@headercolor}
%
%% Para cambiar los colores de la estructura:
%\setbeamercolor{structure}{fg=beamer@headercolor}
%
%%Para cambiar el color del texto de la caja del titulo:
%\setbeamercolor{frametitle}{fg=blue}
%
%%Para cambiar el color del texto y del fondo:
%\definecolor{azultext}{RGB}{43,93,156}
%
%
%%-------------------------------------------------------
%% Definiendo y redefiniendo comandos y entornos
%%-------------------------------------------------------
%
%% colores de los hiperlinks
%\newcommand{\chref}[2]{
%  \href{#1}{{\usebeamercolor[bg]{Feather}#2}}
%}
%%\setbeamertemplate{headline}[text line]{\insertsectionnavigationhorizontal{\paperwidth}{}{}}
%%-------------------------------------------------------
%% Informacioón de la página del titulo
%%-------------------------------------------------------
%%%% configuracion del tema %%%%%%%%%%%%%%%%%%%5
%\makeatletter
%  \definecolor{beamer@barcolor}{cmyk}{0,0.46,0.50,0} % melon  %{RGB}{43,93,156}%azul
%  \definecolor{beamer@headercolor}{HTML}{B0E0E6} % azul claro%{RGB}{226,107,59}%naranja
%\makeatother
%%%%%%%%%%%%%%%%%%%%%%%%%%%%%%%%%%%%%%%%%%%%%%%%%%%%%%%%%%%%%%%%%%%%%%%%%%%%%%%%%%%%%%%%%%%%%%%%%%%
%%%%%%%%%%%%%%%%%%%%%%%%%%%%%%%%%%%%%%%%%%%%%%%%%%%%%%%%%%%%%%%%%%%%%%%%%%%%%%%%%%%%%%%%%%%%%%%%%%%%%
\newcommand{\cdefault}[4][named]{\begin{tikzpicture}
\fill[#2,draw=negro] (0,0) rectangle ++(2,1);
\node[below] at (1,0) {#2};
\node[below=4mm] at (1,0) {\tiny #3 \{#4\}};
\node[below=6mm] at (1,0) {\tiny #1};
\end{tikzpicture}}
%%%%%%%%%%%%%%%%%%%%%%%%%%%%%%%%%%%%%%%%%%%%%%%
\newsavebox\terminalbox
\lstnewenvironment{terminal}[1][]
  {\lstset{#1}\setbox\terminalbox=\vbox\bgroup\hsize=0.7\textwidth}
  {\egroup
   \tikzstyle{terminal} = [
    draw=white, text=white, font=courier, fill=blue!20, very thick,
    rectangle, inner sep=10pt, inner ysep=20pt
   ]
   \tikzstyle{terminalTitle} = [
     fill=red!20, text=white, font=\ttfamily, draw=white
   ]
   \begin{tikzpicture}
   \node [terminal] (box){\usebox{\terminalbox}};
   \node[terminalTitle, rounded corners, right=10pt] at (box.north west) {tty: /bin/bash};
   \end{tikzpicture}
}
%\lstset{escapeinside={/*@}{@*/}}
%%%%%%%%%%------------------------------------%%%%%%%%%%
\newenvironment{slides}[1]
{\begin{frame}[fragile,allowframebreaks, environment=slides]{#1}}
{\end{frame}}

%%%%%%%%%%%-------------------------------%%%%%%%%%%%%%%
%------------------------------------
\titulogrado{Especialista}
\director{Dr: Alejandro Urieles\\ {\footnotesize email: al@gmail}}
\institutedirector{Universidad del Atl\'antico}
\author[Antalcides Olivo]{Antalcides Olivo\\ {\footnotesize email: an@gmail}}
\institute{Universidad del Atl\'antico}
\title{Como realizar tu defensa de tesis con Beamer}
\subtitle{\LaTeX \ una imprenta con estilo}
\date{}
%\subtitle{Linguistics as a Window for Understanding the Brain}
%\titlegraphic{\includegraphics[width=1.5cm]{donald}}

%-------------------------------------------------------
% Cuerpo de la presentación
%-------------------------------------------------------

\begin{document}
%-------------------------------------------------------
% La página del título
%-------------------------------------------------------


\begin{frame}[plain,noframenumbering] % la opción plain elimina la cabecera de la página de título, noframenumbering quita la numeración unicamente de esta diapositiva
  \titlepage % coloca aquí  la información de la página que se estipuló antes.
\end{frame}


\begin{frame}{Contenido}{}
\tableofcontents
\end{frame}


\section{Teoremas}
\begin{frame}{Teoremas}
\begin{teorema}
Macondo era el pueblo de Jos\'e Arcadio Buend\'ia, un habitante con gran imaginaci\'on, casado con \'Ursula Iguar\'an, que sol\'ia comprar inventos a Melquiades, el cabecilla de un grupo de gitanos que aparec\'ian una vez al a\~no con novedosos artilugios. Entre los objetos que le compr\'o hab\'ia un im\'an para buscar oro, una lupa a la cual le pretend\'ia dar aplicaciones militares, mapas portugueses y instrumentos de navegaci\'on. La mayor\'ia de sus experimentos se frustraron, como consecuencia llev\'o a cabo una expedici\'on para conocer otros pueblos, descubri\'o que Macondo estaba rodeada por agua. 
%Los primeros dos hijos de Jos\'e Arcadio y \'Ursula fueron Jos\'e Arcadio, el mayor y Aureliano, el peque\~no. Al a\~no siguiente cuando volvieron los gitanos ya no estaba con ellos Melqu\'iades, que hab\'ia muerto. La novedad que trajeron los gitanos aquel a\~no fue el hielo.
\end{teorema}
\end{frame}
\begin{slides}{Prueba }
\begin{tcolorbox}[breakable,colback=Melon!5,colframe=Melon!75!black,title=My title,enhanced jigsaw,opacityback=0.35]
 \begin{source}{Tcolorbox}{}
 \begin{tcolorbox}[breakable,colback=Melon!5,colframe=Melon!75!black,title=My title,enhanced  jigsaw,opacityback=0.35]
 $\displaystyle\sum\limits_{i=1}^n i = \frac{n(n+1)}{2}$
 \end{tcolorbox}
 \end{source}
\tcblower
  $\displaystyle\sum\limits_{i=1}^n i = \frac{n(n+1)}{2}$
\end{tcolorbox}
\end{slides}
\begin{frame}[fragile]{Entorno}
\begin{source}{Entorno}{c1}
Formas para utilizar cualquier entorno.
\begin{entorno}
 ...
\end{entorno}

% ?Descripción?
\begin{entorno}[(?Descripción?)]
 ...
\end{entorno}

% ?Descripción? + referencia
\begin{entorno}[(? Descripción ?)][referencia]
 ...
\end{entorno}

% Referencia
\begin{entorno}[][referencia] % [] es obligatorio
 ..
\end{entorno}

\end{source}
\end{frame}
\begin{frame}[fragile]{Ejemplo de una definición}
\begin{source}{Entorno definicion}{}
\begin{definicion}
Definimos la multiplicación entre números naturales como una aplicación
\begin{tabular}{cccc}
 $\times$: & $I\!N\times I\!N$ & $\rightarrow$ & %
$I\!N$%
\tabularnewline
 & $\left(n,m\right)$ & $\mapsto$ & $l=\times\left(n,m\right) = nm$\tabularnewline
\end{tabular}, de modo que $\forall m,n\in I\!N$ se cumple 
\begin{description}
\item [{i)}] $\times\left(1,m\right)=\times\left(m,1\right)=m$
\item [{ii)}] $\times\left(\varphi\left(n\right),m\right)=\times\left(n,m\right)+m.$
\end{description}

\end{definicion}
\end{source}
\scalebox{0.7}{\begin{definicion}
Definimos la multiplicación entre números naturales como una aplicación
\begin{tabular}{cccc}
 $\times$: & $I\!N\times I\!N$ & $\rightarrow$ & %
$I\!N$%
\tabularnewline
 & $\left(n,m\right)$ & $\mapsto$ & $l=\times\left(n,m\right) = nm$\tabularnewline
\end{tabular}, de modo que $\forall m,n\in I\!N$ se cumple 
\begin{description}
\item [{i)}] $\times\left(1,m\right)=\times\left(m,1\right)=m$
\item [{ii)}] $\times\left(\varphi\left(n\right),m\right)=\times\left(n,m\right)+m.$
\end{description}

\end{definicion}}
\end{frame}
%%%%%%%%%%%%%%%%%%%%%%%%%%%%%%%%%%%%%%%%%%%%%%%%%%%%%%%%%%%%%%%%%%%%%%%%%%%%%%
\begin{frame}[fragile]{Ejemplo de un teorema} 
\begin{source}{Entorno definicion}{}
\begin{teorema}
?Ningún? ?número? natural coincide con su sucesor, es decir $\forall n\in\na,\: n\neq\varphi\left(n\right).$
\end{teorema}
\end{source}
\scalebox{0.7}{
\begin{teorema}
Ningún número natural coincide con su sucesor, es decir $\forall n\in I\!N,\: n\neq\varphi\left(n\right).$
\end{teorema}
}
\end{frame}
%%%%%%%%%%%%%%%%%%%%%%%%%%%%%%%%%%%%%%%%%%%%%%%%%%%%%%%%%%%%%%%%%%%%%%%%%%%%%%%%%%
\begin{frame}[fragile]{Ejemplo de una proposición}
\begin{source}{Entorno definicion}{}
\begin{proposicion}
Si $\mathfrak{C}$ es una ?colección? de subconjuntos de $\Omega,$ existe una $\sigma-?álgebra?$ minimal que contiene a $\mathfrak{C}$, esto es, existe una $\sigma-?álgebra?$ $\sigma (\mathfrak{C})$ que contiene a $\mathfrak{C}$  tal que si $\mathfrak{B}$ es otra $\sigma-?álgebra?$ que contiene a $\mathfrak{C}$, $\sigma (\mathfrak{C})\subseteq \mathfrak{B}.$
\end{proposicion}
\end{source}
\scalebox{0.7}{
\begin{proposicion}
Si $\mathfrak{C}$ es una colección de subconjuntos de $\Omega,$ existe una $\sigma-\mbox{\'algebra}$ minimal que contiene a $\mathfrak{C}$, esto es, existe una $\sigma-\mbox{\'algebra}$ $\sigma (\mathfrak{C})$ que contiene a $\mathfrak{C}$  tal que si $\mathfrak{B}$ es otra $\sigma-\mbox{\'algebra}$ que contiene a $\mathfrak{C}$, $\sigma (\mathfrak{C})\subseteq \mathfrak{B}.$
\end{proposicion}
}
\end{frame}
%%%%%%%%%%%%%%%%%%%%%%%%%%%%%%%%%%%%%%%%%%%%%%%%%%%%%%%%%%%%%%%%%%%%%%%%%%%%%%%%%%
\begin{frame}[fragile]{Ejemplo de un lema}
\begin{source}{Entorno definicion}{}
\begin{lema}
Sean $f_1, f_2, f_3. \cdots, f$ funciones medibles
\begin{enumerate}
\itemps Si $f_n\<\geq f$ para todo $n$ y $\int fd\mu > -\infty$, \\
$$\liminf_{n\rightarrow \infty} \int f_nd\mu\geq\int\left( \liminf_{n\rightarrow \infty} f_n \right)d\mu  $$
\itemps Si $f_n\geq f$ para todo $n$ y $\int fd\mu < +\infty$, \\
$$\liminf_{n\rightarrow \infty} \int f_nd\mu\leq\int\left( \liminf_{n\rightarrow \infty} f_n \right)d\mu  $$
\end{enumerate}
\end{lema}
\end{source}
\scalebox{0.7}{
\begin{lema}
Sean $f_1, f_2, f_3. \cdots, f$ funciones medibles
\begin{enumerate}
\itemps Si $f_n\<\geq f$ para todo $n$ y $\int fd\mu > -\infty$, \\
$$\liminf_{n\rightarrow \infty} \int f_nd\mu\geq\int\left( \liminf_{n\rightarrow \infty} f_n \right)d\mu  $$
\itemps Si $f_n\geq f$ para todo $n$ y $\int fd\mu < +\infty$, \\
$$\liminf_{n\rightarrow \infty} \int f_nd\mu\leq\int\left( \liminf_{n\rightarrow \infty} f_n \right)d\mu  $$
\end{enumerate}
\end{lema}
}
\end{frame}
%%%%%%%%%%%%%%%%%%%%%%%%%%%%%%%%%%%%%%%%%%%%%%%%%%%%%%%%%%%%%%%%%%%%%%%%%%%%%%%
%%%%%%%%%%%%%%%%%%%%%%%%%%%%%%%%%%%%%%%%%%%%%%%%%%%%%%%%%%%%%%%%%%%%%%%%%%%%%%%%%%
\begin{frame}[fragile]{Ejemplo de un corolario}
\begin{source}{Entorno definicion}{}
\begin{corolario}\\
 Si \hspace{5pt}$\displaystyle \lim_{x\rightarrow c}f_1(x)=L_1, \displaystyle \lim_{x\rightarrow c}f_2(x)=L_2,\dots, \displaystyle \lim_{x\rightarrow c}f_n(x)=L_n $
 entonces
 \begin{enumerate}
 \item[1.] ${\displaystyle \lim_{x\rightarrow c}}[k_1f_1(x)+k_2f_2(x)+\dots+k_nf_n(x)]=k_1L_1+k_2L_2+\dots+k_nL_n $
 \item[2.]${\displaystyle \lim_{x\rightarrow c}}[f_1(x)f_2(x)\dots f_n(x)]=L_1L_2\dots L_n $
 \end{enumerate}
 \end{corolario}
\end{source}
\scalebox{0.7}{
 \begin{corolario}\\
 Si \hspace{5pt}$\displaystyle \lim_{x\rightarrow c}f_1(x)=L_1, \displaystyle \lim_{x\rightarrow c}f_2(x)=L_2,\dots, \displaystyle \lim_{x\rightarrow c}f_n(x)=L_n $
 entonces
 \begin{enumerate}
 \item[1.] \scalebox{0.9}{${\displaystyle \lim_{x\rightarrow c}}[k_1f_1(x)+k_2f_2(x)+\dots+k_nf_n(x)]=k_1L_1+k_2L_2+\dots+k_nL_n $}
 \item[2.]\scalebox{0.9}{${\displaystyle \lim_{x\rightarrow c}}[f_1(x)f_2(x)\dots f_n(x)]=L_1L_2\dots L_n $}
 \end{enumerate}
 \end{corolario}
}
\end{frame}
%%%%%%%%%%%%%%%%%%%%%%%%%%%%%%%%%%%%%%%%%%%%%%%%%%%%%%%%%%%%%%%%%%%%%%%%%%%%%%%%%%%%%
\begin{frame}[fragile]{Presentemos un ejemplo}
\begin{source}{Entorno definicion}{}
\begin{ejemplo}
Demuestre que si $p(x)=a_0+a_1x+\dots + a_nx^n$ entonces
 $$\lim_{x\rightarrow c}p(x)=p(c)$$
 \end{ejemplo}
\end{source}
\scalebox{0.7}{
\begin{ejemplo}
Demuestre que si $p(x)=a_0+a_1x+\dots + a_nx^n$ entonces
 $$\lim_{x\rightarrow c}p(x)=p(c)$$
 \end{ejemplo}
}
\end{frame}
%%%%%%%%%%%%%%%%%%%%%%%%%%%%%%%%%%%%%%%%%%%%%%%%%%%%%%%%%%%%%%%%%%%%%%%%%%%%%%%%%%%%%%%%
\begin{frame}[fragile]{Presentemos una scaja}
\begin{source}{Entorno definicion}{}
\begin{scaja}
Sea $f:\, A\overset{f}{\rightarrow}B$ una ?función? de $A$
en $B$, llamaremos dominio de la ?función? $f$, al conjunto de todas
las primeras componentes, el cual denotaremos por $D_{f}$ , es decir:
\[
D_{f}=\left\{ x\in A:\:\mbox{tal que existe un }y\in B\wedge\left(x,y\right)\in f\right\} \subseteq A.
\]
 \end{scaja}
\end{source}
\scalebox{0.7}{
\begin{scaja}
Sea $f:\, A\overset{f}{\rightarrow}B$ una función de $A$
en $B$, llamaremos dominio de la función $f$, al conjunto de todas
las primeras componentes, el cual denotaremos por $D_{f}$ , es decir:
\[
D_{f}=\left\{ x\in A:\:\mbox{tal que existe un }y\in B\wedge\left(x,y\right)\in f\right\} \subseteq A.
\]
 \end{scaja}
}
\end{frame}
%%%%%%%%%%%%%%%%%%%%%%%%%%%%%%%%%%%%%%%%%%%%%%%%%%%%%%%%%%%%%%%%%%%%%%%%%%%%%%%%%%%%%%%%
\begin{frame}[fragile]{Presentemos el entorno vocabulario}
\begin{source}{Entorno definicion}{}
\begin{vocabulario}
Sea $f:\, A\overset{f}{\rightarrow}B$ una ?función? de $A$
en $B$, llamaremos dominio de la ?función? $f$, al conjunto de todas
las primeras componentes, el cual denotaremos por $D_{f}$ , es decir:
\[
D_{f}=\left\{ x\in A:\:\mbox{tal que existe un }y\in B\wedge\left(x,y\right)\in f\right\} \subseteq A.
\]
 \end{vocabulario}
\end{source}
\scalebox{0.7}{
\begin{vocabulario}
Sea $f:\, A\overset{f}{\rightarrow}B$ una función de $A$
en $B$, llamaremos dominio de la función $f$, al conjunto de todas
las primeras componentes, el cual denotaremos por $D_{f}$ , es decir:
\[
D_{f}=\left\{ x\in A:\:\mbox{tal que existe un }y\in B\wedge\left(x,y\right)\in f\right\} \subseteq A.
\]
 \end{vocabulario}
}
\end{frame}
%%%%%%%%%%%%%%%%%%%%%%%%%%%%%%%%%%%%%%%%%%%%%%%%%%%%%%%%%%%%%%%%%%%%%%%%%%%%%%%%%%%%%%%%
\begin{frame}[fragile]{Presentemos el entorno cajaejercicios}
\begin{source}{Entorno definicion}{}
\begin{vocabulario}
\begin{cajaejercicios}
Sí $p:\:4<7$ y $q:\ 6\,\mbox{ es ?número? par}$.
Calcular el valor de verdad de $p\wedge q.$ 
 \end{cajaejercicios}
\end{source}
\scalebox{0.7}{
\begin{cajaejercicios}
Sí $p:\:4<7$ y $q:\ 6\,\mbox{ es número par}$.
Calcular el valor de verdad de $p\wedge q.$ 
 \end{cajaejercicios}
}
\end{frame}
%%%%%%%%%%%%%%%%%%%%%%%%%%%%%%%%%%%%%%%%%%%%%%%%%%%%%%%%%%%%%%%%%%%%%%%%%%%%%%%%%%%%%%%%
\begin{frame}[fragile]{Presentemos el entorno proof}
\begin{source}{Entorno definicion}{}
\begin{proof}
Para probar que las proposiciones $(p\rightarrow q)$ y
$(\sim q\rightarrow\sim p)$ son ?lógicamente? equivalentes debemos
probar que $(p\rightarrow q)$ $\leftrightarrow$ $(\sim q\rightarrow\sim p)$
es una ?tautología?. 
\end{proof}
\end{source}
\scalebox{0.7}{
\begin{proof}
Para probar que las proposiciones $(p\rightarrow q)$ y
$(\sim q\rightarrow\sim p)$ son lógicamente equivalentes debemos
probar que $(p\rightarrow q)$ $\leftrightarrow$ $(\sim q\rightarrow\sim p)$
es una tautología. 
\end{proof}
}
\end{frame}
%%%%%%%%%%%%%%%%%%%%%%%%%%%%%%%%%%%%%%%%%%%%%%%%%%%%%%%%%%%%%%%%%%%%%%%%%%%%%%%%%%%%%%%%
\begin{frame}[fragile]{Block sin título}
\begin{source}{Entorno definicion}{}
\begin{block}<1->{}
\begin{itemize}
\item Bloque sin ?título?.
\item Mostrado en todas las diapositivas.
\end{itemize}
\end{block}
\end{source}
%\scalebox{0.7}{
\begin{block}<1->{}
\begin{itemize}
\item Bloque sin título.
\item Mostrado en todas las diapositivas.
\end{itemize}
\end{block}
%}
\end{frame}
%%%%%%%%%%%%%%%%%%%%%%%%%%%%%%%%%%%%%%%%%%%%%%%%%%%%%%%%%%%%%%%%%%%%%%%%%%%%%%%%%%%%%%%
\begin{frame}[fragile]{Block de ejemplo con título}
\begin{source}{Entorno definicion}{}
\begin{exampleblock}<2->{?Título? de ?algún? Bloque de ejemplo}
\begin{itemize}
\item $e^{i\pi}=-1$.
\item $e^{i\pi/2}=i$.
\end{itemize}
\end{exampleblock}
\end{source}
%\scalebox{0.7}{
\begin{exampleblock}<2->{Título de algún Bloque de ejemplo}

\begin{itemize}
\item $e^{i\pi}=-1$.
\item $e^{i\pi/2}=i$.
\end{itemize}
\end{exampleblock}
%}
\end{frame}
%%%%%%%%%%%%%%%%%%%%%%%%%%%%%%%%%%%%%%%%%%%%%%%%%%%%%%%%%%%%%%%%%%%%%%%%%%%%%%%%%%%%%%%
\begin{frame}[fragile]{Block example}
\begin{source}{Entorno definicion}{}
\begin{example}
\begin{itemize}
\item $e^{i\pi}=-1$.
\item $e^{i\pi/2}=i$.
\end{itemize}
\end{example}
\end{source}
%\scalebox{0.7}{
\begin{example}
\begin{itemize}
\item $e^{i\pi}=-1$.
\item $e^{i\pi/2}=i$.
\end{itemize}
\end{example}
%}
\end{frame}
%%%%%%%%%%%%%%%%%%%%%%%%%%%%%%%%%%%%%%%%%%%%%%%%%%%%%%%%%%%%%%%%%%%%%%%%%%%%%%%%%%%%%%%
%%%%%%%%%%%%%%%%%%%%%%%%%%%%%%%%%%%%%%%%%%%%%%%%%%%%%%%%%%%%%%%%%%%%%%%%%%%%%%%%%%%%%%%
\begin{frame}[fragile]{Block alertblock}
\begin{source}{Entorno definicion}{}
\begin{alertblock}{Un block para alertas}
\begin{itemize}
\item $e^{i\pi}=-1$.
\item $e^{i\pi/2}=i$.
\end{itemize}
\end{alertblock}
\end{source}
%\scalebox{0.7}{
\begin{alertblock}{Un block para alertas}
\begin{itemize}
\item $e^{i\pi}=-1$.
\item $e^{i\pi/2}=i$.
\end{itemize}
\end{alertblock}
%}
\end{frame}
%%%%%%%%%%%%%%%%%%%%%%%%%%%%%%%%%%%%%%%%%%%%%%%%%%%%%%%%%%%%%%%%%%%%%%%%%%%%%%%%%%%%%%%
\begin{frame}[fragile]{Block theorem}
\begin{source}{Entorno definicion}{}
\begin{theorem}
?Ningún? ?número? natural coincide con su sucesor, es decir $\forall n\in I\!N,\: n\neq\varphi\left(n\right).$
\end{theorem}
\end{source}
%\scalebox{0.7}{
\begin{theorem}
Ningún número natural coincide con su sucesor, es decir $\forall n\in I\!N,\: n\neq\varphi\left(n\right).$
\end{theorem}

%}
\end{frame}
%%%%%%%%%%%%%%%%%%%%%%%%%%%%%%%%%%%%%%%%%%%%%%%%%%%%%%%%%%%%%%%%%%%%%%%%%%%%%%%%%%%%%%%
%%%%%%%%%%%%%%%%%%%%%%%%%%%%%%%%%%%%%%%%%%%%%%%%%%%%%%%%%%%%%%%%%%%%%%%%%%%%%%%%%%%%%%%
\begin{frame}[fragile]{Block corollary}
\begin{source}{Entorno definicion}{}
\begin{corollary}
 Si \hspace{5pt}$\displaystyle \lim_{x\rightarrow c}f_1(x)=L_1, \displaystyle \lim_{x\rightarrow c}f_2(x)=L_2,\dots, \displaystyle \lim_{x\rightarrow c}f_n(x)=L_n $
 entonces
\end{corollary}
\end{source}
%\scalebox{0.7}{
\begin{corollary}
 Si \hspace{5pt}$\displaystyle \lim_{x\rightarrow c}f_1(x)=L_1, \displaystyle \lim_{x\rightarrow c}f_2(x)=L_2,\dots, \displaystyle \lim_{x\rightarrow c}f_n(x)=L_n $
 entonces
\end{corollary}
%}
\end{frame}
%%%%%%%%%%%%%%%%%%%%%%%%%%%%%%%%%%%%%%%%%%%%%%%%%%%%%%%%%%%%%%%%%%%%%%%%%%%%%%%%%%%%%%%
%%%%%%%%%%%%%%%%%%%%%%%%%%%%%%%%%%%%%%%%%%%%%%%%%%%%%%%%%%%%%%%%%%%%%%%%%%%%%%%%%%%%%%%
\begin{frame}[fragile]{Block example}
\begin{source}{Entorno definicion}{}
\begin{example}
 Si \hspace{5pt}$\displaystyle \lim_{x\rightarrow c}f_1(x)=L_1, \displaystyle \lim_{x\rightarrow c}f_2(x)=L_2,\dots, \displaystyle \lim_{x\rightarrow c}f_n(x)=L_n $
 entonces
\end{example}
\end{source}
%\scalebox{0.7}{
\begin{example}
 Si \hspace{5pt}$\displaystyle \lim_{x\rightarrow c}f_1(x)=L_1, \displaystyle \lim_{x\rightarrow c}f_2(x)=L_2,\dots, \displaystyle \lim_{x\rightarrow c}f_n(x)=L_n $
 entonces
\end{example}
%}
\end{frame}

%%%%%%%%%%%%%%%%%%%%%%%%%%%%%%%%%%%%%%%%%%%%%%%%%%%%%%%%%%%%%%%%%%%%%%%%%%%%%%%%%%%%%%%
\begin{frame}[fragile]{Block examples}
\begin{source}{Entorno definicion}{}
\begin{examples}
 Si \hspace{5pt}$\displaystyle \lim_{x\rightarrow c}f_1(x)=L_1, \displaystyle \lim_{x\rightarrow c}f_2(x)=L_2,\dots, \displaystyle \lim_{x\rightarrow c}f_n(x)=L_n $
 entonces
\end{examples}
\end{source}
%\scalebox{0.7}{
\begin{examples}
 Si \hspace{5pt}$\displaystyle \lim_{x\rightarrow c}f_1(x)=L_1, \displaystyle \lim_{x\rightarrow c}f_2(x)=L_2,\dots, \displaystyle \lim_{x\rightarrow c}f_n(x)=L_n $
 entonces
\end{examples}
%}
\end{frame}
%%%%%%%%%%%%%%%%%%%%%%%%%%%%%%%%%%%%%%%%%%%%%%%%%%%%%%%%%%%%%%%%%%%%%%%%%%%%%%%%%%%%%%%
\begin{frame}[fragile]{Block definition}
\begin{source}{Entorno definicion}{}
\begin{definition}
 Si \hspace{5pt}$\displaystyle \lim_{x\rightarrow c}f_1(x)=L_1, \displaystyle \lim_{x\rightarrow c}f_2(x)=L_2,\dots, \displaystyle \lim_{x\rightarrow c}f_n(x)=L_n $
 entonces
\end{definition}
\end{source}
%\scalebox{0.7}{
\begin{definition}
 Si \hspace{5pt}$\displaystyle \lim_{x\rightarrow c}f_1(x)=L_1, \displaystyle \lim_{x\rightarrow c}f_2(x)=L_2,\dots, \displaystyle \lim_{x\rightarrow c}f_n(x)=L_n $
 entonces
\end{definition}
%}
\end{frame}
%%%%%%%%%%%%%%%%%%%%%%%%%%%%%%%%%%%%%%%%%%%%%%%%%%%%%%%%%%%%%%%%%%%%%%%%%%%%%%%%%%%%%%%
\begin{frame}[fragile]{Block fact}
\begin{source}{Entorno definicion}{}
\begin{fact}
 Si \hspace{5pt}$\displaystyle \lim_{x\rightarrow c}f_1(x)=L_1, \displaystyle \lim_{x\rightarrow c}f_2(x)=L_2,\dots, \displaystyle \lim_{x\rightarrow c}f_n(x)=L_n $
 entonces
\end{fact}
\end{source}
%\scalebox{0.7}{
\begin{fact}
 Si \hspace{5pt}$\displaystyle \lim_{x\rightarrow c}f_1(x)=L_1, \displaystyle \lim_{x\rightarrow c}f_2(x)=L_2,\dots, \displaystyle \lim_{x\rightarrow c}f_n(x)=L_n $
 entonces
\end{fact}
%}
\end{frame}
%%%%%%%%%%%%%%%%%%%%%%%%%%%%%%%%%%%%%%%%%%%%%%%%%%%%%%%%%%%%%%%%%%%%%%%%%%%%%%%%%%%%%%%
\begin{frame}[fragile]{Block definitions}
\begin{source}{Entorno definicion}{}
\begin{definitions}
 Si \hspace{5pt}$\displaystyle \lim_{x\rightarrow c}f_1(x)=L_1, \displaystyle \lim_{x\rightarrow c}f_2(x)=L_2,\dots, \displaystyle \lim_{x\rightarrow c}f_n(x)=L_n $
 entonces
\end{definitions}
\end{source}
%\scalebox{0.7}{
\begin{definitions}
 Si \hspace{5pt}$\displaystyle \lim_{x\rightarrow c}f_1(x)=L_1, \displaystyle \lim_{x\rightarrow c}f_2(x)=L_2,\dots, \displaystyle \lim_{x\rightarrow c}f_n(x)=L_n $
 entonces
\end{definitions}
%}
\end{frame}
%%%%%%%%%%%%%%%%%%%%%%%%%%%%%%%%%%%%%%%%%%%%%%%%%%%%%%%%%%%%%%%%%%%%%%%%%%%%%%%%%%%%%%%
\begin{frame}[fragile]{Block definitionT}
\begin{source}{Entorno definicion}{}
\begin{definitionT}
 Si \hspace{5pt}$\displaystyle \lim_{x\rightarrow c}f_1(x)=L_1, \displaystyle \lim_{x\rightarrow c}f_2(x)=L_2,\dots, \displaystyle \lim_{x\rightarrow c}f_n(x)=L_n $
 entonces
\end{definitionT}
\end{source}
%\scalebox{0.7}{
\begin{definitionT}
 Si \hspace{5pt}$\displaystyle \lim_{x\rightarrow c}f_1(x)=L_1, \displaystyle \lim_{x\rightarrow c}f_2(x)=L_2,\dots, \displaystyle \lim_{x\rightarrow c}f_n(x)=L_n $
 entonces
\end{definitionT}
%}
\end{frame}
%%%%%%%%%%%%%%%%%%%%%%%%%%%%%%%%%%%%%%%%%%%%%%%%%%%%%%%%%%%%%%%%%%%%%%%%%%%%%%%%%%%%%%%%%%%%%%%%%%
\begin{frame}[fragile]{Block corollaryT}
\begin{source}{Entorno definicion}{}
\begin{corollaryT}
 Si \hspace{5pt}$\displaystyle \lim_{x\rightarrow c}f_1(x)=L_1, \displaystyle \lim_{x\rightarrow c}f_2(x)=L_2,\dots, \displaystyle \lim_{x\rightarrow c}f_n(x)=L_n $
 entonces
\end{corollaryT}
\end{source}
%\scalebox{0.7}{
\begin{corollaryT}
 Si \hspace{5pt}$\displaystyle \lim_{x\rightarrow c}f_1(x)=L_1, \displaystyle \lim_{x\rightarrow c}f_2(x)=L_2,\dots, \displaystyle \lim_{x\rightarrow c}f_n(x)=L_n $
 entonces
\end{corollaryT}
%}
\end{frame}
%%%%%%%%%%%%%%%%%%%%%%%%%%%%%%%%%%%%%%%%%%%%%%%%%%%%%%%%%%%%%%%%%%%%%%%%%%%%%%%%%%%%%%%%%%%%%%%%%%
\begin{frame}[fragile]{Block ejerciciosT}
\begin{source}{Entorno definicion}{}
\begin{ejerciciosT}
 Si \hspace{5pt}$\displaystyle \lim_{x\rightarrow c}f_1(x)=L_1, \displaystyle \lim_{x\rightarrow c}f_2(x)=L_2,\dots, \displaystyle \lim_{x\rightarrow c}f_n(x)=L_n $
 entonces
\end{ejerciciosT}
\end{source}
%\scalebox{0.7}{
\begin{ejerciciosT}
 Si \hspace{5pt}$\displaystyle \lim_{x\rightarrow c}f_1(x)=L_1, \displaystyle \lim_{x\rightarrow c}f_2(x)=L_2,\dots, \displaystyle \lim_{x\rightarrow c}f_n(x)=L_n $
 entonces
\end{ejerciciosT}
%}
\end{frame}
%%%%%%%%%%%%%%%%%%%%%%%%%%%%%%%%%%%%%%%%%%%%%%%%%%%%%%%%%%%%%%%%%%%%%%%%%%%%%%%%%%%%%%%%%%%%%%%%%%
\begin{frame}[fragile]{Block theoremeT}
\begin{source}{Entorno definicion}{}
\begin{theoremeT}
 Si \hspace{5pt}$\displaystyle \lim_{x\rightarrow c}f_1(x)=L_1, \displaystyle \lim_{x\rightarrow c}f_2(x)=L_2,\dots, \displaystyle \lim_{x\rightarrow c}f_n(x)=L_n $
 entonces
\end{theoremeT}
\end{source}
%\scalebox{0.7}{
\begin{theoremeT}
 Si \hspace{5pt}$\displaystyle \lim_{x\rightarrow c}f_1(x)=L_1, \displaystyle \lim_{x\rightarrow c}f_2(x)=L_2,\dots, \displaystyle \lim_{x\rightarrow c}f_n(x)=L_n $
 entonces
\end{theoremeT}
%}
\end{frame}
%%%%%%%%%%%%%%%%%%%%%%%%%%%%%%%%%%%%%%%%%%%%%%%%%%%%%%%%%%%%%%%%%%%%%%%%%%%%%%%%%%%%%%%%%%%%%%%%%%
\begin{frame}[fragile]{Block Problema}
\begin{source}{Entorno definicion}{}
\begin{problema}
 Si \hspace{5pt}$\displaystyle \lim_{x\rightarrow c}f_1(x)=L_1, \displaystyle \lim_{x\rightarrow c}f_2(x)=L_2,\dots, \displaystyle \lim_{x\rightarrow c}f_n(x)=L_n $
 entonces
\end{problema}
\end{source}
%\scalebox{0.7}{
\begin{problema}
 Si \hspace{5pt}$\displaystyle \lim_{x\rightarrow c}f_1(x)=L_1, \displaystyle \lim_{x\rightarrow c}f_2(x)=L_2,\dots, \displaystyle \lim_{x\rightarrow c}f_n(x)=L_n $
 entonces
\end{problema}
%}
\end{frame}
%%%%%%%%%%%%%%%%%%%%%%%%%%%%%%%%%%%%%%%%%%%%%%%%%%%%%%%%%%%%%%%%%%%%%%%%%%%%%%%%%%%%%%%%%%%%%%%%%%
\begin{frame}[fragile]{Block exerciseT}
\begin{source}{Entorno definicion}{}
\begin{exerciseT}
 Si \hspace{5pt}$\displaystyle \lim_{x\rightarrow c}f_1(x)=L_1, \displaystyle \lim_{x\rightarrow c}f_2(x)=L_2,\dots, \displaystyle \lim_{x\rightarrow c}f_n(x)=L_n $
 entonces
\end{exerciseT}
\end{source}
%\scalebox{0.7}{
\begin{exerciseT}
 Si \hspace{5pt}$\displaystyle \lim_{x\rightarrow c}f_1(x)=L_1, \displaystyle \lim_{x\rightarrow c}f_2(x)=L_2,\dots, \displaystyle \lim_{x\rightarrow c}f_n(x)=L_n $
 entonces
\end{exerciseT}
%}
\end{frame}
%%%%%%%%%%%%%%%%%%%%%%%%%%%%%%%%%%%%%%%%%%%%%%%%%%%%%%%%%%%%%%%%%%%%%%%%%%%%%%%%%%%%%%%%%%%%%%%%%%
\begin{frame}[fragile]{Block ejercicio}
\begin{source}{Entorno definicion}{}
\begin{ejercicio}
 Si \hspace{5pt}$\displaystyle \lim_{x\rightarrow c}f_1(x)=L_1, \displaystyle \lim_{x\rightarrow c}f_2(x)=L_2,\dots, \displaystyle \lim_{x\rightarrow c}f_n(x)=L_n $
 entonces
\end{ejercicio}
\end{source}
%\scalebox{0.7}{
\begin{ejercicio}
 Si \hspace{5pt}$\displaystyle \lim_{x\rightarrow c}f_1(x)=L_1, \displaystyle \lim_{x\rightarrow c}f_2(x)=L_2,\dots, \displaystyle \lim_{x\rightarrow c}f_n(x)=L_n $
 entonces
\end{ejercicio}
%}
\end{frame}
%%%%%%%%%%%%%%%%%%%%%%%%%%%%%%%%%%%%%%%%%%%%%%%%%%%%%%%%%%%%%%%%%%%%%%%%%%%%%%%%%%%%%%%%%%%%%%%%%%
\begin{frame}[fragile]{Block nota}
\begin{source}{Entorno definicion}{}
\begin{nota}
 Si \hspace{5pt}$\displaystyle \lim_{x\rightarrow c}f_1(x)=L_1, \displaystyle \lim_{x\rightarrow c}f_2(x)=L_2,\dots, \displaystyle \lim_{x\rightarrow c}f_n(x)=L_n $
 entonces
\end{nota}
\end{source}
%\scalebox{0.7}{
\begin{nota}
 Si \hspace{5pt}$\displaystyle \lim_{x\rightarrow c}f_1(x)=L_1, \displaystyle \lim_{x\rightarrow c}f_2(x)=L_2,\dots, \displaystyle \lim_{x\rightarrow c}f_n(x)=L_n $
 entonces
\end{nota}
%}
\end{frame}
%%%%%%%%%%%%%%%%%%%%%%%%%%%%%%%%%%%%%%%%%%%%%%%%%%%%%%%%%%%%%%%%%%%%%%%%%%%%%%%%%%%%%%%%%%%%%%%%%%
\begin{frame}[fragile]{Block exer}
\begin{source}{Entorno definicion}{}
\begin{exer}
 Si \hspace{5pt}$\displaystyle \lim_{x\rightarrow c}f_1(x)=L_1, \displaystyle \lim_{x\rightarrow c}f_2(x)=L_2,\dots, \displaystyle \lim_{x\rightarrow c}f_n(x)=L_n $
 entonces
\end{exer}
\end{source}
%\scalebox{0.7}{
\begin{exer}
 Si \hspace{5pt}$\displaystyle \lim_{x\rightarrow c}f_1(x)=L_1, \displaystyle \lim_{x\rightarrow c}f_2(x)=L_2,\dots, \displaystyle \lim_{x\rightarrow c}f_n(x)=L_n $
 entonces
\end{exer}
%}
\end{frame}
%%%%%%%%%%%%%%%%%%%%%%%%%%%%%%%%%%%%%%%%%%%%%%%%%%%%%%%%%%%%%%%%%%%%%%%%%%%%%%%%%%%%%%%%%%%%%%%%%%
\begin{frame}[fragile]{Block ejer}
\begin{source}{Entorno definicion}{}
\begin{ejer}
 Si \hspace{5pt}$\displaystyle \lim_{x\rightarrow c}f_1(x)=L_1, \displaystyle \lim_{x\rightarrow c}f_2(x)=L_2,\dots, \displaystyle \lim_{x\rightarrow c}f_n(x)=L_n $
 entonces
\end{ejer}
\end{source}
%\scalebox{0.7}{
\begin{ejer}
 Si \hspace{5pt}$\displaystyle \lim_{x\rightarrow c}f_1(x)=L_1, \displaystyle \lim_{x\rightarrow c}f_2(x)=L_2,\dots, \displaystyle \lim_{x\rightarrow c}f_n(x)=L_n $
 entonces
\end{ejer}
%}
\end{frame}
%%%%%%%%%%%%%%%%%%%%%%%%%%%%%%%%%%%%%%%%%%%%%%%%%%%%%%%%%%%%%%%%%%%%%%%%%%%%%%%%%%%%%%%%%%%%%%%%%%
\begin{frame}[fragile]{Block solu}
\begin{source}{Entorno definicion}{}
\begin{solu}
 Si \hspace{5pt}$\displaystyle \lim_{x\rightarrow c}f_1(x)=L_1, \displaystyle \lim_{x\rightarrow c}f_2(x)=L_2,\dots, \displaystyle \lim_{x\rightarrow c}f_n(x)=L_n $
 entonces
\end{solu}
\end{source}
%\scalebox{0.7}{
\begin{solu}
 Si \hspace{5pt}$\displaystyle \lim_{x\rightarrow c}f_1(x)=L_1, \displaystyle \lim_{x\rightarrow c}f_2(x)=L_2,\dots, \displaystyle \lim_{x\rightarrow c}f_n(x)=L_n $
 entonces
\end{solu}
%}
\end{frame}
%%%%%%%%%%%%%%%%%%%%%%%%%%%%%%%%%%%%%%%%%%%%%%%%%%%%%%%%%%%%%%%%%%%%%%%%%%%%%%%%%%%%%%%%%%%%%%%%%%
\begin{frame}[fragile]{Block ejercicios}
\begin{source}{Entorno definicion}{}
\begin{definitions}
 Si \hspace{5pt}$\displaystyle \lim_{x\rightarrow c}f_1(x)=L_1, \displaystyle \lim_{x\rightarrow c}f_2(x)=L_2,\dots, \displaystyle \lim_{x\rightarrow c}f_n(x)=L_n $
 entonces
\end{definitions}
\end{source}
%\scalebox{0.7}{
\begin{ejercicios}
 Si \hspace{5pt}$\displaystyle \lim_{x\rightarrow c}f_1(x)=L_1, \displaystyle \lim_{x\rightarrow c}f_2(x)=L_2,\dots, \displaystyle \lim_{x\rightarrow c}f_n(x)=L_n $
 entonces
\end{ejercicios}
%}
\end{frame}
%%%%%%%%%%%%%%%%%%%%%%%%%%%%%%%%%%%%%%%%%%%%%%%%%%%%%%%%%%%%%%%%%%%%%%%%%%%%%%%%%%%%%%%%%%%%%%%%%%
%%%%%%%%%%%%%%%%%%%%%%%%%%%%%%%%%%%%%%%%%%%%%%%%%%%%%%%%%%%%%%%%%%%%%%%%%%%%%%%%%%%%%%%%%%%%%%%%%%%
\section{Ejemplo}

\section{block predefinidos}
\begin{frame}[allowframebreaks]{Prueba 3}
\begin{block}{Block Title}
Lorem ipsum dolor sit amet, consectetur adipisicing elit, sed do eiusmod tempor incididunt ut labore et dolore magna aliqua.
\end{block}
\begin{alertblock}{Block Title}
Lorem ipsum dolor sit amet, consectetur adipisicing elit, sed do eiusmod tempor incididunt ut labore et dolore magna aliqua.
\end{alertblock}
\begin{definition}
A prime number is a number that...
\end{definition}
\begin{example}
Lorem ipsum dolor sit amet, consectetur adipisicing elit, sed do eiusmod tempor incididunt ut labore et dolore magna aliqua.
\end{example} 
\begin{theorem}[Pythagoras] 
$ a^2 + b^2 = c^2$
\end{theorem}
\begin{corollary}
$ x + y = y + x  $
\end{corollary}

\begin{exampleblock}{}
  {\large ``To be, or not to be: that is the question.''}
  \vskip5mm
  \hspace*\fill{\small--- William Shakespeare, Hamlet}
\end{exampleblock}

\end{frame}
%%%%%%%%%%%cambiando el color a los textbullet
\setbeamercolor{itemize item}{fg=red} % all frames will have red bullets

\begin{frame}{cambiando el color de los text bullet}
\setbeamercolor*{item}{fg=red}
  \begin{itemize}
    \item First item.
    \item Second item.
    \item Third item.
    \item Fourth item.
  \end{itemize}

\end{frame}

%%%%%%%%%%%%%%%%%%%%%%%%%%%%%%%%%%%%%%%%%%%%%%%%%%%%%%%%%%%%%%%%%%%%%%%%%%%%%%%%%%%%%%%%%%%%%%%%%%%%
%%%%%%%%%%%%%%%%%%%%%%%%%%%%%%%%%%%%%%%%%%%%%%%%%%%%%%%%%%%%%%%%%%%%%%%%%%%%%%%%%%%%%%%%%%%%%%%%%%%
\section{Bibliografía citada}
\begin{frame}{Material de referencia }

  \begin{thebibliography}{10}

    \bibitem[Kern, 2007]{Kern07}
    Uwe Kern.
    \newblock {\em Extending \LaTeX{}'s color facilities: The {\tt xcolor}
      package}, January 2007.
    
    \bibitem{Matthias10}
    Andreas Matthias.
    \newblock {\em {The pdfpages Package}}, December 2010.
        \bibitem[Tantau, 2008]{Tantau08}
    Till Tantau.
    \newblock {\em {The TikZ and PGF Packages. Manual for version 2.00.}}
    \newblock Institut f\"{u}r Theoretische Informatik, Universit\"{a}t zu
      L\"{u}beck, February 2008.
    
   

  \end{thebibliography}
  
\end{frame}

\end{document}