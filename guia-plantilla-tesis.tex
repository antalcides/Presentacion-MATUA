\batchmode
\makeatletter
\def\input@path{{style/}{sections/}{pdf/}{logos/}}
\makeatother
\documentclass[]{beamer}


%-------------------------------------------------------
% Inclusión de paquetes
%-------------------------------------------------------
%\setbeamertemplate{frametitle continuation}{\frametitle{\color{white}Title}}
\usepackage{lipsum-es}
%\usepackage[latin9]{inputenc}
%\usepackage[ansinew]{inputenc}
%--------------------------------
%
\usepackage{ifthen} 
\usepackage[T1]{fontenc}
\usepackage[mathletters]{ucs}
\usepackage[utf8x]{inputenc}
\usepackage[english,spanish,es-tabla]{babel}
\uselanguage{spanish}
\languagepath{spanish}
\usepackage{environ}
\usepackage{lmodern}
\usefonttheme[onlymath]{serif}
\usefonttheme{professionalfonts}
%\usepackage[spanish]{babel}
\usepackage{times}



%---------------------------------
\usepackage{amssymb,amsfonts,latexsym,cancel,stmaryrd}%
\usepackage[ruled,,vlined,lined,linesnumbered]{algorithm2e}
\usepackage{framed}
\usepackage{mathptmx}
\usepackage{helvet}
%\linespread{1.05}
\usepackage[full]{textcomp}                        % Caracteres especiales como ' (recto)
\usepackage{amsmath}
\makeatletter
\let\@tmp\@xfloat     
\usepackage{fixltx2e}
\let\@xfloat\@tmp                    
\makeatother
\makeatletter
\let\th@plain\relax
\makeatother
\theoremstyle{plain}
\let\openbox\relax
%\usepackage{ntheorem}
\let\proofname\relax
\let\proof\relax
\let\endproof\relax
\usepackage{amsthm}
%\usepackage{ddot}
%\usepackage{a4wide}
\usepackage{amsfonts}
\usepackage{epsfig}
%\usepackage{amscd}
\usepackage{longtable}
%\usepackage{latexcad}
\usepackage{fancybox}
\graphicspath{{ps/}{logos/}{figuras/}{sections/Figures/}}
\listfiles
% COMANDOS PERSONALES   ----------------------------------------------
\newcommand{\R}{\mathbb{R}}
\newcommand{\Z}{\mathbb{Z}}
\newcommand{\Q}{\mathbb{Q}}
\newcommand{\N}{\mathbb{N}}
\newcommand{\I}{\mathbb{I}}
\newcommand{\raya}{\rule{2cm}{0.01cm}\\}
\newcommand{\ds}{\displaystyle}
\newcommand{\sen}{\mathop{\rm sen}\nolimits}
\newcommand{\senh}{\mathop{\rm senh}\nolimits}
\newcommand{\arcsen}{\mathop{\rm arcsen}\nolimits}
\newcommand{\arcsec}{\mathop{\rm arcsec}\nolimits}
%\def\sen{\mathop{\mbox{\normalfont sen}}\nolimits}
\def\max{\mathop{\mbox{\normalfont m\’ax}}}
\newcommand{\bc}{\begin{center}}
\newcommand{\ec}{\end{center}}
\newcommand{\be}{\begin{enumerate}}
\newcommand{\ee}{\end{enumerate}}
%----------- cargando el Tema Tesis
\DeclareGraphicsExtensions{.eps,.jpg,.png, .bmp}
%%%%%%%%%%%%% colocando pagina de presentación para una charla
 \newcommand\makebeamertitle{\frame{\hacetitle}}%

\usetheme[]{Tesis}%%% opción para el tema outher.
%    progressstyle=fixedCircCnt,   % fixedCircCnt, movingCircCnt (Movimiento por defecto)
  
% Si desea cambiar los colores de los diversos elementos en el tema, edite y descomente las siguientes líneas

% Para cambiar los colores de la barra:
%\setbeamercolor{Tesis}{fg=red!20,bg=red}

% Para cambiar los colores de la estructura:
%\setbeamercolor{structure}{fg=red}

%Para cambiar el color del texto de la caja del titulo:
%\setbeamercolor{frametitle}{fg=blue}

%Para cambiar el color del texto y del fondo:
%\definecolor{azultext}{RGB}{43,93,156}

%-------------------------------------------------------
% Definiendo y redefiniendo comandos y entornos
%-------------------------------------------------------

% colores de los hiperlinks
\newcommand{\chref}[2]{
  \href{#1}{{\usebeamercolor[bg]{Feather}#2}}
}
%\setbeamertemplate{headline}[text line]{\insertsectionnavigationhorizontal{\paperwidth}{}{}}
%-------------------------------------------------------
% Informacioón de la página del titulo
%-------------------------------------------------------
%%% configuracion del tema %%%%%%%%%%%%%%%%%%%5
\makeatletter
  \definecolor{beamer@barcolor}{RGB}{21,46,128}%azul
  \definecolor{beamer@headercolor}{RGB}{226,107,59}%naranja
\makeatother
\newcommand{\cdefault}[4][named]{\begin{tikzpicture}
\fill[#2,draw=negro] (0,0) rectangle ++(2,1);
\node[below] at (1,0) {#2};
\node[below=4mm] at (1,0) {\tiny #3 \{#4\}};
\node[below=6mm] at (1,0) {\tiny #1};
\end{tikzpicture}}
%%%%%%%%%%%%%%%%%%%%%%%%%%%%%%%%%%%%%%%%%%%%%%%
\newsavebox\terminalbox
\lstnewenvironment{terminal}[1][]
  {\lstset{#1}\setbox\terminalbox=\vbox\bgroup\hsize=0.7\textwidth}
  {\egroup
   \tikzstyle{terminal} = [
    draw=white, text=white, font=courier, fill=blue!20, very thick,
    rectangle, inner sep=10pt, inner ysep=20pt
   ]
   \tikzstyle{terminalTitle} = [
     fill=red!20, text=white, font=\ttfamily, draw=white
   ]
   \begin{tikzpicture}
   \node [terminal] (box){\usebox{\terminalbox}};
   \node[terminalTitle, rounded corners, right=10pt] at (box.north west) {tty: /bin/bash};
   \end{tikzpicture}
}
%%%%%%%%%%------------------------------------%%%%%%%%%%
\newenvironment{slides}[1]
{\begin{frame}[fragile,allowframebreaks, environment=slides]{#1}}
{\end{frame}}

%%%%%%%%%%%-------------------------------%%%%%%%%%%%%%%
%------------------------------------
%\titulo{Especialista}
%\director{Dr: Alejandro Urieles\\ {\footnotesize email: al@gmail}}
%\institutedirector{Universidad del Atl\'antico}
%\author[Antalcides Olivo]{Antalcides Olivo\\ {\footnotesize email: an@gmail}}
%\institute{Universidad del Atl\'antico}
%\title{Como realizar tu defensa de tesis con Beamer}
%\subtitle{\LaTeX \ una imprenta con estilo}
%\date{}
%\subtitle{Linguistics as a Window for Understanding the Brain}
%\titlegraphic{\includegraphics[width=1.5cm]{donald}}

%-------------------------------------------------------
% Cuerpo de la presentación
%-------------------------------------------------------

\begin{document}
%-------------------------------------------------------
% La página del título
%-------------------------------------------------------
\titulo{Software libre en la enseñanza de las matemáticas}


\subtitulo{Nuevas tecnologías}


\autor{Antalcides Olivo B\inst{1}}


\institute{\inst{1}antalcides@gmail.com}


\date{Mayo del 2013}
\titulographic{\includegraphics[width=2.5cm]{logos/2}}

\titulo[Web 2 ]{Software libre en la enseñanza de las matemáticas}

\begin{frame}[plain,noframenumbering] % la opción plain elimina la cabecera de la página de título, noframenumbering quita la numeración unicamente de esta diapositiva
 \titulopage % coloca aquí  la información de la página que se estipuló antes.
\end{frame}


\begin{frame}{Contenido}{}
\tableofcontents
\end{frame}

\section{Planteamiento del problema}
\begin{slides}{Planteamiento del problema}
Planteamiento del problema....
\end{slides}
\section{ Marco te\'orico}
\begin{slides}{Marco te\'orico}
Resumen del marco teorico....
\end{slides}
\section{ Metodología}
\begin{slides}{Metodología}
Metodología...
\end{slides}
\section{Resultados y Discusión}
\begin{slides}{Resultados y discusión}
Resultados....
\end{slides}
\section{Conclusiones}
\begin{slides}{Conclusiones}
Conclusiones....
\end{slides}
\section{Recomendaciones}
\begin{slides}{Recomendaciones}
Recomendaciones....
\end{slides}
\section{Teoremas}
\begin{frame}{Teoremas}
\begin{teorema}
Macondo era el pueblo de Jos\'e Arcadio Buend\'ia, un habitante con gran imaginaci\'on, casado con \'Ursula Iguar\'an, que sol\'ia comprar inventos a Melquiades, el cabecilla de un grupo de gitanos que aparec\'ian una vez al a\~no con novedosos artilugios. Entre los objetos que le compr\'o hab\'ia un im\'an para buscar oro, una lupa a la cual le pretend\'ia dar aplicaciones militares, mapas portugueses y instrumentos de navegaci\'on. La mayor\'ia de sus experimentos se frustraron, como consecuencia llev\'o a cabo una expedici\'on para conocer otros pueblos, descubri\'o que Macondo estaba rodeada por agua. 
%Los primeros dos hijos de Jos\'e Arcadio y \'Ursula fueron Jos\'e Arcadio, el mayor y Aureliano, el peque\~no. Al a\~no siguiente cuando volvieron los gitanos ya no estaba con ellos Melqu\'iades, que hab\'ia muerto. La novedad que trajeron los gitanos aquel a\~no fue el hielo.
\end{teorema}
\end{frame}
\begin{frame}{Prueba 1}
\begin{tcolorbox}[breakable,colback=blue!5,colframe=blue!75!black,title=My title]
  My cool formalization
\tcblower
  $\displaystyle\sum\limits_{i=1}^n i = \frac{n(n+1)}{2}$
\end{tcolorbox}
\end{frame}
\begin{frame}[fragile]{Entorno}
\begin{source}{Entorno}{c1}
\begin{entorno}
 ...
\end{entorno}

%Descripción
\begin{entorno}[(Descripción)]
 ...
\end{entorno}

%Descripción + referencia
\begin{entorno}[(Descripción)][referencia]
 ...
\end{entorno}

%Referencia
\begin{entorno}[][referencia] %[] es manadatorio
 ..
\end{entorno}

\end{source}
\end{frame}
\section{Bibliografía citada}
\begin{frame}{Material de referencia }

  \begin{thebibliography}{10}

    \bibitem[Kern, 2007]{Kern07}
    Uwe Kern.
    \newblock {\em Extending \LaTeX{}'s color facilities: The {\tt xcolor}
      package}, January 2007.
    
    \bibitem{Matthias10}
    Andreas Matthias.
    \newblock {\em {The pdfpages Package}}, December 2010.
        \bibitem[Tantau, 2008]{Tantau08}
    Till Tantau.
    \newblock {\em {The TikZ and PGF Packages. Manual for version 2.00.}}
    \newblock Institut f\"{u}r Theoretische Informatik, Universit\"{a}t zu
      L\"{u}beck, February 2008.
    
   

  \end{thebibliography}
  
\end{frame}

\end{document}