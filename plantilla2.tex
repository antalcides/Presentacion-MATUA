\batchmode
\makeatletter
\def\input@path{{style/}{sections/}{pdf/}{logos/}}
\makeatother
\documentclass[]{beamer}


%-------------------------------------------------------
% Inclusi\'{o}n de paquetes
%-------------------------------------------------------
%\setbeamertemplate{frametitle continuation}{\frametitle{\color{white}Title}}
\usepackage{lipsum}
%\usepackage[latin9]{inputenc}
%\usepackage[ansinew]{inputenc}
%--------------------------------
%
\usepackage{ifthen}
\usepackage[T1]{fontenc}
\usepackage[mathletters]{ucs}
\usepackage[utf8x]{inputenc}
\usepackage[english,spanish,es-tabla]{babel}
\uselanguage{spanish}
\languagepath{spanish}
\usepackage{environ}
\usepackage{lmodern}
\usefonttheme[onlymath]{serif}
\usefonttheme{professionalfonts}
%\usepackage[spanish]{babel}
\usepackage{times}



%---------------------------------
\usepackage{amssymb,amsfonts,latexsym,cancel,stmaryrd}%
\usepackage[ruled,,vlined,lined,linesnumbered]{algorithm2e}
\usepackage{framed}
\usepackage{mathptmx}
\usepackage{helvet}
%\linespread{1.05}
\usepackage[full]{textcomp}                        % Caracteres especiales como ' (recto)
\usepackage{amsmath}
\makeatletter
\let\@tmp\@xfloat
\usepackage{fixltx2e}
\let\@xfloat\@tmp
\makeatother
\makeatletter
\let\th@plain\relax
\makeatother
\theoremstyle{plain}
\let\openbox\relax
%\usepackage{ntheorem}
\let\proofname\relax
\let\proof\relax
\let\endproof\relax
\usepackage{amsthm}
%\usepackage{ddot}
%\usepackage{a4wide}
\usepackage{amsfonts}
\usepackage{epsfig}
%\usepackage{amscd}
\usepackage{longtable}
%\usepackage{latexcad}
\usepackage{fancybox}
\graphicspath{{ps/}{logos/}{figuras/}{sections/Figures/}}
\listfiles
% COMANDOS PERSONALES   ----------------------------------------------
\newcommand{\R}{\mathbb{R}}
\newcommand{\Z}{\mathbb{Z}}
\newcommand{\Q}{\mathbb{Q}}
\newcommand{\N}{\mathbb{N}}
\newcommand{\I}{\mathbb{I}}
\newcommand{\raya}{\rule{2cm}{0.01cm}\\}
\newcommand{\ds}{\displaystyle}
\newcommand{\sen}{\mathop{\rm sen}\nolimits}
\newcommand{\senh}{\mathop{\rm senh}\nolimits}
\newcommand{\arcsen}{\mathop{\rm arcsen}\nolimits}
\newcommand{\arcsec}{\mathop{\rm arcsec}\nolimits}
%\def\sen{\mathop{\mbox{\normalfont sen}}\nolimits}
\def\max{\mathop{\mbox{\normalfont m\’ax}}}
\newcommand{\bc}{\begin{center}}
\newcommand{\ec}{\end{center}}
\newcommand{\be}{\begin{enumerate}}
\newcommand{\ee}{\end{enumerate}}
%----------- cargando el Tema Tesis
\DeclareGraphicsExtensions{.eps,.jpg,.png, .bmp}


\usetheme[]{Tesis}%%% opci\'{o}n para el tema outher.
%    progressstyle=fixedCircCnt,   % fixedCircCnt, movingCircCnt (Movimiento por defecto)

% Si desea cambiar los colores de los diversos elementos en el tema, edite y descomente las siguientes l\'{\i}neas

% Para cambiar los colores de la barra:
%\setbeamercolor{Tesis}{fg=red!20,bg=red}

% Para cambiar los colores de la estructura:
%\setbeamercolor{structure}{fg=red}

%Para cambiar el color del texto de la caja del titulo:
%\setbeamercolor{frametitle}{fg=blue}

%Para cambiar el color del texto y del fondo:
%\definecolor{azultext}{RGB}{43,93,156}

%-------------------------------------------------------
% Definiendo y redefiniendo comandos y entornos
%-------------------------------------------------------

% colores de los hiperlinks
\newcommand{\chref}[2]{
  \href{#1}{{\usebeamercolor[bg]{Feather}#2}}
}
%\setbeamertemplate{headline}[text line]{\insertsectionnavigationhorizontal{\paperwidth}{}{}}
%-------------------------------------------------------
% Informacio\'{o}n de la p\'{a}gina del titulo
%-------------------------------------------------------
%%% configuracion del tema %%%%%%%%%%%%%%%%%%%5
\makeatletter
  \definecolor{beamer@barcolor}{RGB}{21,46,128}%azul
  \definecolor{beamer@headercolor}{RGB}{226,107,59}%naranja
\makeatother
\newcommand{\cdefault}[4][named]{\begin{tikzpicture}
\fill[#2,draw=negro] (0,0) rectangle ++(2,1);
\node[below] at (1,0) {#2};
\node[below=4mm] at (1,0) {\tiny #3 \{#4\}};
\node[below=6mm] at (1,0) {\tiny #1};
\end{tikzpicture}}
%%%%%%%%%%%%%%%%%%%%%%%%%%%%%%%%%%%%%%%%%%%%%%%
\newsavebox\terminalbox
\lstnewenvironment{terminal}[1][]
  {\lstset{#1}\setbox\terminalbox=\vbox\bgroup\hsize=0.7\textwidth}
  {\egroup
   \tikzstyle{terminal} = [
    draw=white, text=white, font=courier, fill=blue!20, very thick,
    rectangle, inner sep=10pt, inner ysep=20pt
   ]
   \tikzstyle{terminalTitle} = [
     fill=red!20, text=white, font=\ttfamily, draw=white
   ]
   \begin{tikzpicture}
   \node [terminal] (box){\usebox{\terminalbox}};
   \node[terminalTitle, rounded corners, right=10pt] at (box.north west) {tty: /bin/bash};
   \end{tikzpicture}
}
%\lstset{escapeinside={/*@}{@*/}}
%%%%%%%%%%------------------------------------%%%%%%%%%%
\newenvironment{slides}[1]
{\begin{frame}[fragile,allowframebreaks, environment=slides]{#1}}
{\end{frame}}

%%%%%%%%%%%-------------------------------%%%%%%%%%%%%%%
%------------------------------------
\titulo{Especialista}
\director{Dr: Alejandro Urieles\\ {\footnotesize email: al@gmail}}
\institutedirector{Universidad del Atl\'antico}
\author[Antalcides Olivo]{Antalcides Olivo\\ {\footnotesize email: an@gmail}}
\institute{Universidad del Atl\'antico}
\title{Como realizar tu defensa de tesis con Beamer}
\subtitle{\LaTeX \ una imprenta con estilo}
\date{}
%\subtitle{Linguistics as a Window for Understanding the Brain}
%\titlegraphic{\includegraphics[width=1.5cm]{donald}}

%-------------------------------------------------------
% Cuerpo de la presentaci\'{o}n
%-------------------------------------------------------

\begin{document}
%-------------------------------------------------------
% La p\'{a}gina del t\'{\i}tulo
%-------------------------------------------------------


\begin{frame}[plain,noframenumbering] % la opci\'{o}n plain elimina la cabecera de la p\'{a}gina de t\'{\i}tulo, noframenumbering quita la numeraci\'{o}n unicamente de esta diapositiva
  \titlepage % coloca aqu\'{\i}  la informaci\'{o}n de la p\'{a}gina que se estipul\'{o} antes.
\end{frame}


\begin{frame}{Contenido}{}
\tableofcontents
\end{frame}


\section{Teoremas}
\begin{frame}{Teoremas}
\begin{teorema}
Macondo era el pueblo de Jos\'e Arcadio Buend\'ia, un habitante con gran imaginaci\'on, casado con \'Ursula Iguar\'an, que sol\'ia comprar inventos a Melquiades, el cabecilla de un grupo de gitanos que aparec\'ian una vez al a\~no con novedosos artilugios. Entre los objetos que le compr\'o hab\'ia un im\'an para buscar oro, una lupa a la cual le pretend\'ia dar aplicaciones militares, mapas portugueses y instrumentos de navegaci\'on. La mayor\'ia de sus experimentos se frustraron, como consecuencia llev\'o a cabo una expedici\'on para conocer otros pueblos, descubri\'o que Macondo estaba rodeada por agua.
%Los primeros dos hijos de Jos\'e Arcadio y \'Ursula fueron Jos\'e Arcadio, el mayor y Aureliano, el peque\~no. Al a\~no siguiente cuando volvieron los gitanos ya no estaba con ellos Melqu\'iades, que hab\'ia muerto. La novedad que trajeron los gitanos aquel a\~no fue el hielo.
\end{teorema}
\end{frame}
\begin{frame}{Prueba 1}
\begin{tcolorbox}[breakable,colback=blue!5,colframe=blue!75!black,title=My title]
  My cool formalization
\tcblower
  $\displaystyle\sum\limits_{i=1}^n i = \frac{n(n+1)}{2}$
\end{tcolorbox}
\end{frame}
\begin{frame}[fragile]{Entorno}
\begin{source}{Entorno}{c1}
Formas para utilizar cualquier entorno.
\begin{entorno}
 ...
\end{entorno}

% ?Descripci\'{o}n?
\begin{entorno}[(?Descripci\'{o}n?)]
 ...
\end{entorno}

% ?Descripci\'{o}n? + referencia
\begin{entorno}[(? Descripci\'{o}n ?)][referencia]
 ...
\end{entorno}

% Referencia
\begin{entorno}[][referencia] % [] es obligatorio
 ..
\end{entorno}

\end{source}
\end{frame}
\section{Ejemplo}

\section{block predefinidos}
\begin{frame}[allowframebreaks]{Prueba 3}
\begin{block}{Block Title}
Lorem ipsum dolor sit amet, consectetur adipisicing elit, sed do eiusmod tempor incididunt ut labore et dolore magna aliqua.
\end{block}
\begin{alertblock}{Block Title}
Lorem ipsum dolor sit amet, consectetur adipisicing elit, sed do eiusmod tempor incididunt ut labore et dolore magna aliqua.
\end{alertblock}
\begin{definition}
A prime number is a number that...
\end{definition}
\begin{example}
Lorem ipsum dolor sit amet, consectetur adipisicing elit, sed do eiusmod tempor incididunt ut labore et dolore magna aliqua.
\end{example}
\begin{theorem}[Pythagoras]
$ a^2 + b^2 = c^2$
\end{theorem}
\begin{corollary}
$ x + y = y + x  $
\end{corollary}

\begin{exampleblock}{}
  {\large ``To be, or not to be: that is the question.''}
  \vskip5mm
  \hspace*\fill{\small--- William Shakespeare, Hamlet}
\end{exampleblock}

\end{frame}
%%%%%%%%%%%cambiando el color a los textbullet
\setbeamercolor{itemize item}{fg=red} % all frames will have red bullets

\begin{frame}{cambiando el color de los text bullet}
\setbeamercolor*{item}{fg=red}
  \begin{itemize}
    \item First item.
    \item Second item.
    \item Third item.
    \item Fourth item.
  \end{itemize}

\end{frame}

\section{Bibliograf\'{\i}a citada}
\begin{frame}{Material de referencia }

  \begin{thebibliography}{10}

    \bibitem[Kern, 2007]{Kern07}
    Uwe Kern.
    \newblock {\em Extending \LaTeX{}'s color facilities: The {\tt xcolor}
      package}, January 2007.

    \bibitem{Matthias10}
    Andreas Matthias.
    \newblock {\em {The pdfpages Package}}, December 2010.
        \bibitem[Tantau, 2008]{Tantau08}
    Till Tantau.
    \newblock {\em {The TikZ and PGF Packages. Manual for version 2.00.}}
    \newblock Institut f\"{u}r Theoretische Informatik, Universit\"{a}t zu
      L\"{u}beck, February 2008.



  \end{thebibliography}

\end{frame}

\end{document} 